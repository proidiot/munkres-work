\documentclass[main.tex]{subfiles}

\begin{document}

\subproblem{}\label{2e}
\[A \setminus (A \setminus B) = B\]
\begin{thm}[Equality fails in~\ref{2e}]
	Given two sets \(A\) and \(B\), it is not the case that \(B\) is equal
	to the portion of \(A\) removing the other portion of \(A\) which does
	not overlap with \(B\). Symbolically, that is:
	\[\neg [A \setminus (A \setminus B) = B]\]
\end{thm}
\begin{proof}
	For equality among sets to hold, it is necessary that both sets are
	demonstrably subsets of each other. Specifically, given particular
	choices for sets \(A\) and \(B\), we will explore whether \(B\) is a
	subset of \(A \setminus (A \setminus B)\).

	Choose set \(A\) to be the set of all even integers. Choose set \(B\) to
	be the set of all odd integers. Since the set of all even integers does
	not have any members that are also in the set of all odd integers,
	removing \(B\) from \(A\) still leaves the whole of \(A\). That is, for
	our particular choices for \(A\) and \(B\), we have that
	\(A \setminus B = A\). By substitution, removing from \(A\) the portion
	of \(A\) that does not overlap with \(B\) is simply removing all of
	\(A\) from itself. Removing such a set from itself must result in the
	empty set.

	Continuing with our previous choices for sets \(A\) and \(B\), suppose
	it really was the case that \(B\) is a subset of the portion of \(A\)
	removing the other portion of \(A\) which does not overlap with \(B\).
	Since we have previously demonstrated that
	\(A \setminus (A \setminus B)\) is the empty set, that would mean that
	the set of all odd integers is a subset of the empty set. However, that
	is absurd, as there certainly do exist odd integers, and that would
	contradict the notion that \(B\) is a subset of the empty set. Since we
	have reached a contradiction, we can say that we were incorrect in our
	supposition, and so it is not always the case that \(B\) is a subset of
	the portion of \(A\) removing the other portion of \(A\) which does not
	overlap with \(B\).
\end{proof}

\begin{thm}[LHS is a subset of RHS in~\ref{2e}]
	Given two sets \(A\) and \(B\), it is the case that \(B\) is a subset of
	the portion of \(A\) removing the other portion of \(A\) which does not
	overlap with \(B\). Symbolically, that is:
	\[A \setminus (A \setminus B) \subset B\]
\end{thm}
\begin{proof}
	Assume there are two arbitrary sets \(A\) and \(B\). Note that removing
	the portion of \(A\) which overlaps with \(B\) is equivalent to taking
	the intesection of \(A\) and the complement of \(B\). Now removing from
	\(A\) the intersection of \(A\) and the complement of \(B\) leaves only
	the intersection of \(A\) and \(B\). Finally, it is the case that the
	intersection of \(A\) and \(B\) is a subset of \(B\) given the
	definition of set intersection.
\end{proof}

\end{document}

