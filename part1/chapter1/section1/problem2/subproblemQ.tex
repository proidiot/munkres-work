\documentclass[main.tex]{subfiles}

\begin{document}

\subproblem{}\label{2q}
\[(A \times B) \setminus (C \times D) = (A \setminus C) \times (B \setminus D)\]
\begin{thm}[Equality fails in~\ref{2q}]
	Given sets of \(A\), \(B\), \(C\), and \(D\), it is not the case that
	the portion of the cartesian product of \(A\) with \(B\) wihtout the
	portion of the cartesian product of \(C\) with \(D\) is equal to the
	cartesian product of the portion of \(A\) which does not overlap with
	\(C\) with the portion of \(B\) which does not overlap with \(D\).
	Symbolically, that is
	\[\neg [(A \times B) \setminus (C \times D) = (A \setminus C) \times (B \setminus D)]\]
\end{thm}
\begin{proof}
	Choose \(A\) to be the set \(\{1\}\), choose \(B\) to be the set
	\(\{2\}\), choose \(C\) to also be the set \(\{1\}\), and choose \(D\)
	to be the empty set. So \(A \setminus C\) is the empty set and
	\(B \setminus D\) is the same as \(B\). Now constructing some cartesian
	products, we have that \(A \times B\) is \(\{(1,2)\}\), that
	\(C \times D\) is the empty set, and that
	\((A \setminus C) \times (B \setminus D)\) is also the empty set. By the
	definition of set difference, we can see that
	\((A \times B) \setminus (C \times D)\) is \(\{(1,2)\}\). Since
	\(\{(1,2)\}\) is not empty, our choices have demonstrated that it is not
	always the case that \((A \times B) \setminus (C \times D)\) is equal to
	\((A \setminus C) \times (B \setminus D)\).
\end{proof}

\begin{thm}[RHS is a subset of LHS in~\ref{2q}]
	Given sets of \(A\), \(B\), \(C\), and \(D\), the cartesian product of
	the portion of \(A\) which does not overlap with \(C\) with the portion
	of \(B\) which does not overlap with \(D\) is a subset of the portion of
	the cartesian product of \(A\) with \(B\) wihtout the portion of the
	cartesian product of \(C\) with \(D\). Symbolically, that is
	\[(A \setminus C) \times (B \setminus D) \subset (A \times B) \setminus (C \times D)\]
\end{thm}
\begin{proof}
	Assume \(A\), \(B\), \(C\), and \(D\) are sets. Since
	\((A \setminus C) \times (B \setminus D)\) being empty immediately
	demonstrates that it would be a subset of
	\((A \times B) \setminus (C \times D)\), we will instead consider
	whether some arbitrary value in the set would also imply that such a
	value would also be in \((A \times B) \setminus (C \times D)\). So let
	\((x,y)\) be in \((A \setminus C) \times (B \setminus D)\). We know that
	\(x\) must be in \(A \setminus C\) and \(y\) must be in
	\(B \setminus D\) given the nature of the cartesian product. Now the
	definition of set difference gives us that \(x\) is in \(A\), that \(x\)
	is not in \(C\), that \(y\) is in \(B\), and that \(y\) is not in \(D\).
	We can construct \(A \times B\), which will have \((x,y)\) as a member
	given the definition of the cartesian product. We can also construct
	\(C \times D\), but we need to establish whether or not it contains
	\((x,y)\).

	Suppose \((x,y)\) is in \(C \times D\). Then \(x\) is in \(C\) and \(y\)
	is in \(D\). However this is absurd, as we have already shown that \(x\)
	is not in \(C\) (without even going into the matter of the nature of
	\(y\)). So it must be the case that \((x,y)\) is not in \(C \times D\),
	and thus we also know that \((x,y)\) is in
	\((A \times B) \setminus (C \times D)\) by definition.

	Therefore \((A \setminus C) \times (B \setminus D)\) is a subset of
	\((A \times B) \setminus (C \times D)\).
\end{proof}

\end{document}

