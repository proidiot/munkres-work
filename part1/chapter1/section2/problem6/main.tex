\makeatletter
\def\input@path{{../}}
\makeatother
\documentclass[../main.tex]{subfiles}

\begin{document}

\problem{}\label{s2p6}

Let \(\mapping{f}{\reals}{\reals}\) be the function \(f(x) = x^3 - x\). By
restricting the domain and range of \(f\) appropriately, obtain from \(f\) a
bijective function \(g\). Draw the graphs of \(g\) and \(\inverse{g}\). (There
are several possible choices for \(g\).)

\begin{thm}
	Let \(\mapping{f}{\reals}{\reals}\) be defined by \(f(x) = x^3 - x\).
	Given \(A\) and \(B\) are closed intervals on \(\reals\) such that
	\(A = [-1/\sqrt{3},1/\sqrt{3}]\) and \(B = [-2/3\sqrt{3},2/3\sqrt{3}]\),
	it is the case that \(\mapping{g}{A}{B}\) defined by the restriction of
	\(f\) to \(A\) is bijective.
\end{thm}
\begin{proof}
	Let \(f\) be a mapping from \(\reals\) to \(\reals\) defined by
	\(f(x) = x^3 - x\), let \(A\) be the set consisting of the closed
	interval of \(\reals\) between \(-1/\sqrt{3}\) and \(1/\sqrt{3}\),
	let \(B\) be the set consisting of the closed interval of \(\reals\)
	between \(-2/{3\sqrt{3}}\) and \(2/{3\sqrt{3}}\), and let \(g\) be a
	mapping from \(A\) to \(B\) defined by the restriction of \(f\) to
	\(A\).

	First, we will determine if \(g\) is injective. \todo{}

	Next, we will determine if \(g\) is surjective. \todo{}

	As we have demonstrated that \(g\) is both injective and surjective, we
	have therefore shown that \(g\) is bijective by definition.
\end{proof}

\begin{figure}[H]
\centering{}
\resizebox {\textwidth} {!} {
\begin{tikzpicture}[baseline]
	\begin{axis}[
		xlabel={\(x\)},
		ylabel={\(y\)},
		axis lines=middle,
		xmin=-0.66, xmax=0.66,
		ymin=-0.66, ymax=0.66,
		minor y tick num=0.2,
	]
		\addplot
		[black, domain=-1/sqrt(3):1/sqrt(3), samples=200]
			{x^3 - x}
		node[below right, pos=0]
			{\(g(x)\)};
	\end{axis}
\end{tikzpicture} %
\hskip 10 pt %
\begin{tikzpicture}[baseline]
	\begin{axis}[
		xlabel={\(x\)},
		ylabel={\(y\)},
		axis lines=middle,
		xmin=-0.66, xmax=0.66,
		ymin=-0.66, ymax=0.66,
		minor y tick num=0.2,
	]
		\addplot
		[black, domain=-1/sqrt(3):1/sqrt(3), samples=200]
			({x^3 - x},{x})
		node[below left, pos=1]
			{\(\inverse{g}(x)\)};
	\end{axis}
\end{tikzpicture}
}
\caption{The bijective function \(g\) defined by the restriction of \(f\) to the closed interval bounded by \(\pm 1/\sqrt{3}\), and the inverse of \(g\).}
\end{figure}

\end{document}
