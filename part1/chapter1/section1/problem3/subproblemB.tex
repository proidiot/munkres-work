\documentclass[main.tex]{subfiles}

\begin{document}

\subproblem{}\label{3b}
If \(x > 0\), then \(x^2 - x > 0\).

\begin{thm}[Implication fails in~\ref{3b}]
	Given \(x\) is a postive real number, it is not necessarilly the case
	that \(x^2 - x\) is a positive number. Symbolically, that is
	\[\neg (\forall x \in \mathbb{R}: x > 0 \implies x^2 - x > 0)\]
\end{thm}
\begin{proof}
	Choose \(x\) to be \(\frac{1}{2}\), which is a positive real number. Now
	\(x^2\) is \(\frac{1}{4}\), and \(x^2 - x\) is \(-\frac{1}{4}\). Since
	\(-\frac{1}{4}\) is not a positive number, having \(x\) be a positive
	real number is insufficient for \(x^2 - x\) to be a positive number.
\end{proof}

\begin{remark}[Contrapositive of~\ref{3b}]
	The statement
	\[x^2 -x \leq 0 \implies x \leq 0\]
	is the contrapositive of the statement
	\[x > 0 \implies x^2 - x > 0\]
\end{remark}
\begin{thm}[Implication fails for the contrapositive of~\ref{3b}]
	Given \(x\) is a real number such that \(x^2 - x\) is a non-positive
	number, it is not necessarilly the case that \(x\) is a non-positive
	number.
\end{thm}
\begin{proof}
	Let \(x'\) be \(\frac{1}{2}\). Now \(x'^2\) is \(\frac{1}{4}\), and
	\(x'^2 - x'\) is \(-\frac{1}{4}\). Note that \(-\frac{1}{4}\) is a
	non-positive number. So we can choose \(x\) to be \(x'\) since \(x'\)
	is indeed a real number that would yield a non-positive value for
	\(x^2 - x\). However, note that \(\frac{1}{2}\) is a positive number,
	and so it is not a non-positive number. Therefore having a real number
	\(x\) such that \(x^2 - x\) is a non-positive number is insufficient for
	\(x\) to also be a non-positive number.
\end{proof}

\begin{remark}[Converse of~\ref{3b}]
	The statement
	\[x^2 - x > 0 \implies x > 0\]
	is the converse of the statement
	\[x > 0 \implies x^2 - x > 0\]
\end{remark}
\begin{thm}[Implication fails for the converse of~\ref{3b}]
	Given \(x\) is a real number such that \(x^2 - x\) is a positive number,
	it is not necessarilly the case that \(x\) is also a positive number.
	Symbolically, that is
	\[\neg (\forall x \in \mathbb{R}: x^2 - x > 0 \implies x > 0)\]
\end{thm}
\begin{proof}
	Let \(x'\) be \(-1\), which is a real number. Now \(x'^2\) is 1, and
	\(x'^2 - x'\) is 2. Since 2 is a positive number, we can choose \(x\) to
	be \(x'\). However, since \(x'\) is not a positive number, we have that
	\(x\) is not a positive number.
\end{proof}

\end{document}
