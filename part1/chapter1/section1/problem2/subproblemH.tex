\documentclass[main.tex]{subfiles}

\begin{document}

\subproblem{}\label{2h}
\[A \cup (B \setminus C) = (A \cup B) \setminus (A \cup C)\]
\begin{thm}[Equality fails in~\ref{2h}]
	Given sets \(A\), \(B\), and \(C\), it is not the case that the union of
	\(A\) with the portion of \(B\) which does not overlap with \(C\) is
	equal to the set resulting from removing the union of \(A\) and \(C\)
	from the union of \(A\) and \(B\). Symbolically, that is
	\[\neg [A \cup (B \setminus C) = (A \cup B) \setminus (A \cup C)]\]
\end{thm}
\begin{proof}
	Choose \(A\) to be the set of multiples of 3, \(B\) to be the set of
	multiples of 4, and \(C\) to be the set of multiples of 2. We can see
	that \(B \setminus C\) is the empty set since all multiples of 4 are
	also multiples of 2. So \(A \cup (B \setminus C)\) is just the multiples
	of 3 given the behavior of set union with the empty set. Note that
	\(A \cup (B \setminus C)\) is not empty since there really do exist
	multiples of 3 (for example, 3 itself).

	Given our choices, \(A \cup B\) is the set of all multiples of 3 or 4,
	and \(A \cup C\) is the set of all multiples of 3 or 2. Since any
	multiple of 3 or 4 (or both) would certainly be a multiple of either 3
	or 2 or both, then \((A \cup B) \setminus (A \cup C)\) is the empty set.

	Since \(A \cup (B \setminus C)\) is a non-empty set and
	\((A \cup B) \setminus (A \cup C)\) is the empty set, it is certainly
	not the case that \(A \cup (B \setminus C)\) is a subset of
	\((A \cup B) \setminus (A \cup C)\). Therefore, we can say that equality
	of the sets is not possible in this case.
\end{proof}

\begin{thm}[RHS is a subset of LHS in~\ref{2h}]
	Given sets \(A\), \(B\), and \(C\), the set resulting from removing the
	union of \(A\) and \(C\) from the union of \(A\) and \(B\) is a subset
	of the union of \(A\) and the portion of \(B\) which does not overlap
	with \(C\). Symbolically, that is
	\[(A \cup B) \setminus (A \cup C) \subset A \cup (B \setminus C)\]
\end{thm}
\begin{proof}
	Let \(A\), \(B\), and \(C\) be sets. Since the empty set is already
	known to be a subset of all sets, we will simply examine whether some
	arbitrary value which might exist in \((A \cup B) \setminus (A \cup C)\)
	would necessarilly also be in \(A \cup (B \setminus C)\).

	Let \(x\) be some arbitrary member of
	\((A \cup B) \setminus (A \cup C)\). We know that \(x\) is in
	\(A \cup B\) and that \(x\) is not in \(A \cup C\). Using the definition
	of set union, we can say both that \(x\) is in \(A\) or \(B\) and that
	it is not the case that \(x\) is in \(A\) or \(C\) (or both). De
	Morgan's Law tells use that the latter is equivalent to the statement
	that both \(x\) is not in \(A\) and \(x\) is not in \(C\). This combined
	with the previously established statement that \(x\) is in \(A\) or
	\(x\) is in \(B\) (or both), the disjunctive syllogism informs us that
	it must be that \(x\) is in \(B\). Taking this together with \(x\) not
	being in \(C\), by definition we can say \(x\) is in \(B \setminus C\).
	It follows that \(x\) is in \(A \cup (B \setminus C)\) by disjunctive
	introduction.

	So \((A \cup B) \setminus (A \cup C)\) is a subset of
	\(A \cup (B \setminus C)\).
\end{proof}

\end{document}

