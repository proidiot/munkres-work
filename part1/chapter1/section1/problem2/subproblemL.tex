\documentclass[main.tex]{subfiles}

\begin{document}

\subproblem{}\label{2l}
The converse of~\ref{2j}, assuming that \(A\) and \(B\) are nonempty.
\begin{remark}[The converse of~\ref{2j} with nonempty \(A\) and \(B\)]
	The statement
	\[A \neq \emptyset \land B \neq \emptyset \land (A \times B) \subset (C \times D) \implies A \subset C \land B \subset D\]
	is the converse of
	\[A \subset C \land B \subset D \implies (A \times B) \subset (C \times D)\]
	except with the added restriction that neither \(A\) nor \(B\) are empty.
\end{remark}

\begin{thm}[Implication holds in~\ref{2l}]
	Given sets \(A\), \(B\), \(C\), and \(D\), if \(A\) and \(B\) are
	nonempty and the cartesian product of \(A\) and \(B\) is a subset of the
	cartesian product of \(C\) and \(D\), then both \(A\) is a subset of \(C\)
	and \(B\) is a subset of \(D\). Symbolically, that is
	\[A \neq \emptyset \land B \neq \emptyset \land (A \times B) \subset (C \times D) \implies A \subset C \land B \subset D\]
\end{thm}
\begin{proof}
	Assume \(A\), \(B\), \(C\), and \(D\) are sets such that neither \(A\)
	nor \(B\) are the empty set and the cartesian product of \(A\) and \(B\)
	is a subset of the cartesian product of \(C\) and \(D\). Since neither
	\(A\) nor \(B\) are the empty set, we can say that the cartesian product
	of \(A\) and \(B\) is also not the empty set. Let \((x, y)\) be in
	\(A \times B\) where \(x\) is in \(A\) and \(y\) is in \(B\). Since we
	have established that \(A \times B\) is a subset of \(C \times D\), we
	know that \((x, y)\) is also in \(C \times D\). Given the definition of
	the cartesian product, we can say that both \(x\) is in \(C\) and \(y\)
	is in \(D\). As previously stated, \(x\) is an arbitrary member of \(A\)
	and \(y\) is an arbitrary member of \(B\), so we can say that \(A\) is a
	subset of \(C\) and \(B\) is a subset of \(D\).
\end{proof}

\end{document}

