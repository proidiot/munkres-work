\documentclass[main.tex]{subfiles}

\begin{document}
\problem{Check the distributive laws for \(\sunion\) and \(\sintersect\) and De Morgan's laws.}

\begin{thm}[\(\sintersect\) distributes over \(\cup\)]
	Set intersection distributes across set union. That is, for any sets
	\(A\), \(B\), and \(C\),
	\(A \sintersect (B \sunion C) = (A \sintersect B) \sunion (A \sintersect C)\)
\end{thm}
\begin{proof}
	Given that sets are said to be equal iff they are subsets of each other,
	the goal can be shown directly by demonstrating both of the sets in
	question are subsets of each other. Symbolically, that would be:
	\begin{itemize}
		\item \([A \sintersect (B \sunion C)] \subset [(A \sintersect B) \sunion (A \sintersect C)]\)
		\item \([(A \sintersect B) \sunion (A \sintersect C)] \subset [A \sintersect (B \sunion C)]\)
	\end{itemize}

	\medskip
	First, let's consider the nature of \(A \sintersect (B \sunion C)\): either the
	set is empty, or there exists at least one element in the set. We will
	address these possibilities independently.

	If \(A \sintersect (B \sunion C) = \emptyset\), then
	\([A \sintersect (B \sunion C)] \subset [(A \sintersect B) \sunion (A \sintersect C)]\) since the
	empty set is a subset of all sets.

	If \(\exists x \in A \sintersect (B \sunion C)\), then both \(x \in A\) and
	\(x \in B \sunion C\) by the definition of set intersection. Further, by
	the definition of set union, \(x \in B\) or \(x \in C\). Without loss of
	generality, assume \(x \in B\). Since both \(x \in A\) and \(x \in B\),
	the definition of set intersection gives us \(x \in A \sintersect B\). By
	disjunctive introduction, \(x \in A \sintersect B \lor x \in A \sintersect C\), and
	by the definition of set union, \(x \in (A \sintersect B) \sunion (A \sintersect C)\). So
	\(\exists x \in A \sintersect (B \sunion C) \implies x \in (A \sintersect B) \sunion (A \sintersect C)\),
	and by the definition of subset
	\([A \sintersect (B \sunion C)] \subset [(A \sintersect B) \sunion (A \sintersect C)]\).

	Thus, regardless if \(A \sintersect (B \sunion C) = \emptyset\) or
	\(\exists x \in A \sintersect (B \sunion C)\), it is certain that
	\([A \sintersect (B \sunion C)] \subset [(A \sintersect B) \sunion (A \sintersect C)]\).

	\medskip
	Second, let's consider the nature of \((A \sintersect B) \sunion (A \sintersect C)\):
	either the set is empty, or there exists at least one element in the
	set. We will address these possibilities independently.

	If \((A \sintersect B) \sunion (A \sintersect C) = \emptyset\), then
	\([(A \sintersect B) \sunion (A \sintersect C)] \subset [A \sintersect (B \sunion C)]\) since the
	empty set is a subset of all sets.

	If \(\exists x \in (A \sintersect B) \sunion (A \sintersect C)\), then either
	\(x \in A \sintersect B\) or \(x \in A \sintersect C\) by the definition of set union.
	Without loss of generality, assume \(x \in A \sintersect B\). By the definition
	of set intersection, \(x \in A\) and \(x \in B\). By disjunctive
	introduction, \(x \in B \lor x \in C\), and by the definition of set
	union, \(x \in B \sunion C\). By the definition of set intersection,
	\(x \in A \sintersect (B \sunion C)\). So
	\(\exists x \in (A \sintersect B) \sunion (A \sintersect C) \implies x \in A \sintersect (B \sunion C)\),
	and by definition of subset
	\([(A \sintersect B) \sunion (A \sintersect C)] \subset [A \sintersect (B \sunion C)]\).

	Thus, regardless if \((A \sintersect B) \sunion (A \sintersect C) = \emptyset\) or
	\(\exists x \in (A \sintersect B) \sunion (A \sintersect C)\), it is certain that
	\([(A \sintersect B) \sunion (A \sintersect C)] \subset [A \sintersect (B \sunion C)]\).

	\medskip
	Since both
	\([A \sintersect (B \sunion C)] \subset [(A \sintersect B) \sunion (A \sintersect C)]\) and
	\([(A \sintersect B) \sunion (A \sintersect C)] \subset [A \sintersect (B \sunion C)]\) it is
	therefore the case that
	\(A \sintersect (B \sunion C) = (A \sintersect B) \sunion (A \sintersect C)\).
\end{proof}

\bigbreak{}

\begin{thm}[\(\cup\) distributes over \(\cap\)]
	Set union distributes across set intersection. That is, given sets
	\(A\), \(B\), and \(C\),
	\(A \cup (B \cap C) = (A \cup B) \cap (A \cup C)\)
\end{thm}
\begin{proof}
	Given the definition of set equality, it is sufficent to demonstrate
	equality by showing that each set is a subset of the other. Symbolically:
	\begin{itemize}
		\item \([A \cup (B \cap C)] \subset [(A \cup B) \cap (A \cup C)]\)
		\item \([(A \cup B) \cap (A \cup C)] \subset [A \cup (B \cap C)]\)
	\end{itemize}

	\medskip
	First, let's consider the nature of \(A \cup (B \cap C)\). In the
	trivial case that the set has no memebers (i.e., it is equal to the
	empty set), then
	\([A \cup (B \cap C)] \subset [(A \cup B) \cap (A \cup C)]\) since the
	empty set is a subset of all other sets. If instead
	\(\exists x \in A \cup (B \cap C)\), then either \(x \in A\) or
	\(x \in (B \cap C)\) (or both) by the definition of set union.

	When \(x \in A\), then \(x \in A \lor x \in B\) and
	\(x \in A \lor x \in C\) by disjunctive introduction, which gives
	\(x \in A \cup B\) and \(x \in A \cup C\) by the definition of set
	union.By the definition of set intersection,
	\(x \in [(A \cup B) \cap (A \cup C)]\). So assuming \(x \in A\) gives
	\([A \cup (B \cap C)] \subset [(A \cup B) \cap (A \cup C)]\).

	When \(x \in (B \cap C)\), then \(x \in B\) and \(x \in C\) by the
	definition of set intersection. This gives both \(x \in A \cup B\) and
	\(x \in A \cup C\) by disjunctive introduction and the definition of set
	union. By the definition of set intersection, we get that
	\(x \in (A \cup B) \cap (A \cup C)\). So assuming \(x \in (B \cap C)\)
	gives \([A \cup (B \cap C)] \subset [(A \cup B) \cap (A \cup C)]\).

	So \(\exists x \in A \cup (B \cap C)\) also gives
	\([A \cup (B \cap C)] \subset [(A \cup B) \cap (A \cup C)]\) regardless
	of the particular conditions when \(x \in A \cup (B \cap C)\). Thus it
	is true that \([A \cup (B \cap C)] \subset [(A \cup B) \cap (A \cup C)]\)
	in all circumstances.

	\medskip
	Second, let's consider the nature of \((A \cup B) \cap (A \cup C)\). In
	the trivial case that the set has no members (i.e., it is equal to the
	empty set), then
	\([(A \cup B) \cap (A \cup C)] \subset [A \cup (B \cap C)]\) since the
	empty set is a subset of all other sets. However, there is more to
	consider if the set is non-empty.

	When \((A \cup B) \cap (A \cup C) \neq \emptyset\), then
	\(\exists x \in (A \cup B) \cap (A \cup C)\) by the negation of the
	definition of the empty set. By the definition of set intersection,
	\(x \in A \cup B \land x \in A \cup C\). By the definition of set union,
	\((x \in A \lor x \in B) \land (x \in A \lor x \in C)\). We can proceed
	by separately considering the cases that \(x \in A\),
	\(x \notin A \land x \in B\), and \(x \notin A \land x \in C\). Note
	that either \(x \notin A \land x \notin B\) or
	\(x \notin A \land x \notin C\) would directly contradict
	\((x \in A \lor x \in B) \land (x \in A \lor x \in C)\).

	If \(x \in A\), then \((x \in A) \lor (x \in B \cap C)\) by disjunctive
	introduction. By the definition of set union,
	\(x \in A \cup (B \cap C)\).

	Next, let's say \(x \notin A \land x \in B\). Now it would also seem
	either \(x \in C\) or \(x \notin C\).

	Suppose \(x \notin C\). With what is given, we would directly have that
	\(x \notin A \land x \notin C\). But, as noted above, this contradicts
	what has already been established. So it is not the case that
	\(x \notin C\).

	So we have reached \(x \notin A \land x \in B \land x \in C\). By the
	definition of set intersection, we have
	\(x \notin A \land x \in B \cap C\). By conjunctive elimination, we have
	\(x \in B \cap C\). By disjunctive introduction, we have
	\((x \in A) \lor (x \in B \cap C)\). By the definition of set union, we
	have \(x \in A \cup (B \cap C)\).

	The demonstration from \(x \notin A \land x \in C\) is equivalently
	constructed as the demonstration from \(x \notin A \land x \in B\).

	Since all cases descending from
	\(\exists x \in (A \cup B) \cap (A \cup C)\) lead either to
	contradiction or a demonstration that \(x \in A \cup (B \cap C)\), it is
	certain that \((A \cup B) \cap (A \cup C) \neq \emptyset\) implies
	\((A \cup B) \cap (A \cup C) \subset A \cup (B \cap C)\). Thus, when
	taken with the trivial case that was previously shown, it has been
	demonstrated that
	\((A \cup B) \cap (A \cup C) \subset A \cup (B \cap C)\).

	\medskip
	Therefore, since the relevant sets have been shown to be subsets of each
	other, by the definition of set equality we have that
	\(A \cup (B \cap C) = (A \cup B) \cap (A \cup C)\).
\end{proof}

\bigbreak{}

\begin{thm}[De Morgan's First Law]
	Set difference across set union is equivalent to set intersection of set
	differences. That is, given sets \(A\), \(B\), and \(C\),
	\(A \setminus (B \cup C) = (A \setminus B) \cap (A \setminus C)\)
\end{thm}
\begin{proof}
	Given the definition of set equality, it is sufficent to demonstrate
	equality by showing that each set is a subset of the other. Symbolically:
	\begin{itemize}
		\item \([A \setminus (B \cup C)] \subset [(A \setminus B) \cap (A \setminus C)]\)
		\item \([(A \setminus B) \cap (A \setminus C)] \subset [A \setminus (B \cup C)]\)
	\end{itemize}

	\medskip
	First, let's consider the nature of \(A \setminus (B \cup C)\). In the
	trivial case that the set has no memebers (i.e., it is equal to the
	empty set), then
	\([A \setminus (B \cup C)] \subset [(A \setminus B) \cap (A \setminus C)]\)
	since the empty set is a subset of all other sets. If instead
	\(\exists x \in A \setminus (B \cup C)\), then it is certain that
	\(x \in A\) and \(x \notin (B \cup C)\) by the definition of set
	difference. We can proceed by exploring whether or not \(x \in B\) or
	\(x \in C\).

	Suppose \(x \in B\). By disjunctive introduction,
	\((x \in B) \lor (x \in C)\). By the definition of set union,
	\(x \in (B \cup C)\). But this contradicts \(x \notin (B \cup C)\),
	which has already been shown from our assumptions. So our supposition
	is false, and it is certain that \(x \notin B\). A similar consturction
	starting with a supposition of \(x \in C\) yields a similar
	contradiction, so we also have that \(x \notin C\).

	To recap, we have that \(x \in A\), \(x \notin (B \cup C)\),
	\(x \notin B\), and \(x \notin C\). So we have both
	\((x \in A) \land (x \notin B)\) and \((x \in A) \land (x \notin C)\).
	By the definition of set difference, we have both
	\(x \in (A \setminus B)\) and \(x \in (A \setminus C)\). By the
	definition of set intersection, we have that
	\(x \in (A \setminus B) \cap (A \setminus C)\). Thus we have shown
	\(A \setminus (B \cup C) \subset (A \setminus B) \cap (A \setminus C)\)
	in all circumstances.

	\medskip
	Second, let's consider the nature of
	\((A \setminus B) \cap (A \setminus C)\). In the trivial case that the
	set has no memebers (i.e., it is equal to the empty set), then
	\((A \setminus B) \cap (A \setminus C) \subset A \setminus (B \cup C)\)
	since the empty set is a subset of all other sets. If instead
	\(\exists x \in (A \setminus B) \cap (A \setminus C)\), then it is
	certain that both \(x \in (A \setminus B)\) and
	\(x \in (A \setminus C)\) by the definition of set intersection. By the
	definition of set difference, we have \((x \in A) \land (x \notin B)\)
	and \((x \in A) \land (x \notin C)\). So we have \(x \in A\),
	\(x \notin B\), and \(x \notin C\).

	Suppose \(x \in (B \cup C)\). Then either \(x \in B\) or \(x \in C\) or
	both by the definition of set union. But each of these possibilities
	contradicts what we have already established, so the supposition is
	false and \(x \notin (B \cup C)\).

	Since we have both \(x \in A\) and \(x \notin (B \cup C)\), then we have
	\(x \in A \setminus (B \cup C)\) by the definition of set difference.
	Thus we have shown
	\((A \setminus B) \cap (A \setminus C) \subset A \setminus (B \cup C)\)
	in all circumstances.

	\medskip
	Since we have demonstrated each set is a subset of the other, we have
	therefore demonstrated that
	\(A \setminus (B \cup C) = (A \setminus B) \cap (A \setminus C)\).
\end{proof}

\bigbreak{}

\begin{thm}[De Morgan's Second Law]
	Set difference across set intersection is equivalent to set union of set
	differences. That is, given sets \(A\), \(B\), and \(C\),
	\(A \setminus (B \cap C) = (A \setminus B) \cup (A \setminus C)\)
\end{thm}
\begin{proof}
	Given the definition of set equality, it is sufficent to demonstrate
	equality by showing that each set is a subset of the other. Symbolically:
	\begin{itemize}
		\item \([A \setminus (B \cap C)] \subset [(A \setminus B) \cup (A \setminus C)]\)
		\item \([(A \setminus B) \cup (A \setminus C)] \subset [A \setminus (B \cap C)]\)
	\end{itemize}

	\medskip
	First, let's consider the nature of \(A \setminus (B \cap C)\). In the
	trivial case that the set has no members (i.e., it is equal to the
	empty set), then
	\([A \setminus (B \cap C)] \subset [(A \setminus B) \cup (A \setminus C)]\)
	since the empty set is a subset of all other sets. If instead
	\(\exists x \in A \setminus (B \cap C)\), then \(x \in A\) and
	\(x \notin (B \cap C)\) by the definition of set difference. We can
	proceed by examining whether or not \(x \in B\) and \(x \in C\).

	If \(x \notin B\), then combining with the already established fact that
	\(x \in A\) gives \(x \in A \setminus B\) by the definition of set
	intersection. By disjunctive introduction, we have that
	\((x \in A \setminus B) \lor (x \in A \setminus C)\). By the definition
	of set union, we have that
	\(x \in (A \setminus B) \cup (A \setminus C)\). So
	\(\exists x \in A \setminus (B \cap C) \land x \notin B\) is sufficent
	to demonstrate that
	\(A \setminus (B \cap C) \subset (A \setminus B) \cup (A \setminus C)\).
	A similar construction assuming \(x \notin C\) yields
	\(\exists x \in A \setminus (B \cap C) \land x \notin C \implies
	x \in (A \setminus B) \cup (A \setminus C)\).

	The previous cases admit the possibility that the alternate set in
	question may or may not contain member \(x\) (i.e., the previous
	demonstration in the case that \(x \notin B\) holds for both \(x \in C\)
	and \(x \notin C\)). However, suppose both \(x \in B\) and \(x \in C\).
	Then by the definition of set intersection, \(x \in (B \cap C)\). But
	this contradicts \(x \notin (B \cap C)\), which has already been
	established directly from our assumptions. So it is not the case that
	both \(x \in B\) and \(x \in C\).

	Since all cases eminating from \(\exists x \in A \setminus (B \cap C)\)
	yield either contradictions or demonstrations that
	\(x \in (A \setminus B) \cup (A \setminus C)\), and since the case that
	\(A \setminus (B \cap C) = \emptyset\) has also given the particular
	result, we can say that
	\(A \setminus (B \cap C) \subset (A \setminus B) \cup (A \setminus C)\)
	in all circumstances.

	\medskip
	Second, let's consider the nature of
	\((A \setminus B) \cup (A \setminus C)\). In the trivial case that the
	set has no members (i.e., it is equal to the empty set), then
	\((A \setminus B) \cup (A \setminus C) \subset A \setminus (B \cap C)\)
	since the empty set is a subset of all other sets. If instead
	\(\exists x \in (A \setminus B) \cup (A \setminus C)\), then either
	\(x \in A \setminus B\) or \(x \in A \setminus C\) or both by the
	definition of set union.

	If \(x \in A \setminus B\), then \(x \in A\) and \(x \notin B\) by the
	definition of set difference. Suppose \(x \in (B \cap C)\). Then
	\(x \in B\) and \(x \in C\) by the definition of set intersection. But
	this contradicts that \(x \notin B\), which has already been established
	by the assumptions we have made for this case. So
	\(x \notin (B \cap C)\). Since we have already established that
	\(x \in A\), we have that \(x \in A \setminus (B \cap C)\) by the
	definition of set difference. So
	\(\exists x \in (A \setminus B) \cup (A \setminus C) \land x \in (A \setminus B)
	\implies x \in A \setminus (B \cap C)\). A similar construction
	demonstrates that the case when \(x \in A \setminus C\) also yields
	\(\exists x \in (A \setminus B) \cup (A \setminus C) \land x \in (A \setminus C)
	\implies x \in A \setminus (B \cap C)\). Neither of these cases
	precludes the possibility that both \(x \in A \setminus B\) and
	\(x \in A \setminus C\), but the possibility that neither is true is
	precluded by the assumption
	\(\exists x \in (A \setminus B) \cup (A \setminus C)\).

	Since all cases eminating from
	\(\exists x \in (A \setminus B) \cup (A \setminus C)\) yield either
	contradictions or demonstrations that \(x \in A \setminus (B \cap C)\),
	and since the case that
	\((A \setminus B) \cup (A \setminus C) = \emptyset\) has also given the
	following result, we can say that
	\((A \setminus B) \cup (A \setminus C) \subset A \setminus (B \cap C)\)
	in all circumstances.

	\medskip
	Since we have demonstrated each set is a subset of the other, we have
	therefore demonstrated that
	\(A \setminus (B \cap C) = (A \setminus B) \cup (A \setminus C)\).
\end{proof}
\end{document}
