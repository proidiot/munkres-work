\documentclass[main.tex]{subfiles}

\begin{document}

\subproblem{}\label{s03p04b}

Let \(\theset{A^*}\) be the set of equivalence classes. Show there is a
bijective correspondence of \(\theset{A^*}\) with \(\theset{B}\).

\begin{thm}\thlabel{s03p04bt1}
	An equivalence class must be non-empty.
\end{thm}
\begin{proof}
	Let \(\therelation{\sim}\) be an equivalence relation on \(\theset{A}\),
	and let \(\theset{E_x}\) be an equivalence class for an element
	\(x \in \theset{A}\) on that relation. By the definition of an
	equivalence relation, we know that \(\therelation{\sim}\) is reflexive.
	So \(x \sim x\). Thus \(x \in \theset{E_x}\). Therefore \(\theset{E_x}\)
	is non-empty.
\end{proof}

\begin{thm}\thlabel{s03p04bt2}
	Given \(\therelation{=}\) is an equivalence relation on \(\theset{B}\),
	a surjective function \(\mapping{f}{\theset{A}}{\theset{B}}\), and
	\(\therelation{\sim}\) is a relation on \(\theset{A}\) such that, for
	\(\thepair{a_0}{a_1} \in {\theset{A}^2}\), \({a_0} \sim {a_1}\) if and
	only if \(f(a_0) = f(a_1)\), then \(\therelation{\sim}\) is an
	equivalence relation whose set of equivalence classes \(\theset{A^*}\)
	has a bijective correspondence with \(\theset{B}\).
\end{thm}
\begin{proof}
	Let \(\therelation{=}\) be an equivalence relation on \(\theset{B}\).
	Let \(\mapping{f}{\theset{A}}{\theset{B}}\) be a surjective function.
	Let \(\therelation{\sim}\) be a relation on \(\theset{A}\) such that,
	for \(\thepair{a_0}{a_1} \in {\theset{A}^2}\), \({a_0} \sim {a_1}\) iff
	\(f(a_0) = f(a_1)\). By \thref{s03p04at1}, we know that
	\(\therelation{\sim}\) is an equivalence relation. Let \(\theset{A^*}\)
	be the corresponding set of equivalence classes. Let
	\(\mapping{g}{\theset{B}}{\theset{A^*}}\) such that \(g(y)\) gives the
	equivalence class of \(x \in \theset{A}\) where \(y = f(x)\). We will
	show that \(g\) is a bijective correspondence between \(\theset{B}\)
	and \(\theset{A^*}\) by showing things.

	First, let \({y_0} \in \theset{B}\) and \({y_1} \in \theset{B}\) such
	that \({y_0} \neq {y_1}\). Since \(\mapping{f}{\theset{A}}{\theset{B}}\)
	is surjective, we know that there must exist values
	\({x_0} \in \theset{A}\) and \({x_1} \in \theset{A}\) such that
	\({y_0} = f(x_0)\) and \({y_1} = f(x_1)\). Since we can see that
	\(f(x_0) \neq f(x_1)\) by substitution, the rule of assignment gives us
	that \({x_0} \neq {x_1}\). Suppose that \(g(y_0) = g(y_1)\). By
	substitution, we would have that \(g(f(x_0)) = g(f(x_1))\). By the
	definition of \(g\), we would have that the equivalence class of \(x_0\)
	would be equal to the equivalence class of \(x_1\). By the definition of
	\(\therelation{\sim}\), that would give us that \(f(x_0) = f(x_1)\),
	which contradicts \(f(x_0) \neq f(x_1)\), which was already established.
	So the supposition is false, and \(g(y_0) \neq g(y_1)\). Thus \(g\) is
	injective.

	Next, let \(\theset{E} \in \theset{A^*}\). By the definition of
	\(\theset{A^*}\), \(\theset{E}\) is an equivalence class of
	\(\therelation{\sim}\). By \thref{s03p04bt1}, \(\theset{E}\) is
	non-empty. Let \(x \in \theset{E}\). By the definition of an equivalence
	class of \(\therelation{\sim}\), we know there is some
	\(y \in \theset{B}\) such that \(y = f(x)\). By the definition of \(g\),
	we have that \(g(y) = \theset{E}\). Since an arbitrary
	\(\theset{E} \in \theset{A^*}\) necessarilly has a corresponding
	\(y \in \theset{B}\) such that \(g(y) = \theset{E}\), we have thus shown
	that \(g\) is surjective.

	Since \(g\) is both injective and surjective, we have therefore
	demonstrated that there is a bijective correspondence between
	\(\theset{A^*}\) and \(\theset{B}\).
\end{proof}

\end{document}
