\makeatletter
\def\input@path{{../}}
\makeatother
\documentclass[../main.tex]{subfiles}

\begin{document}

\problem{}\label{s3p3}

Here is a ``proof'' that every relation \(C\) that is both symmetric and
transitive is also reflexive: ``Since \(C\) is symmetric, \(aCb\) implies
\(bCa\). Since \(C\) is transitive, \(aCb\) and \(bCa\) together imply \(aCa\),
as desired.'' Find the flaw in this argument.

\begin{remark}[Symmetry and Transitivity are Insufficient for Reflexivity]
	It is true that given \(\therelation{C}\) is a symmetric and transitive
	relation on \(\theset{A}\), we can say that any
	\(\thepair{a}{b} \in \therelation{C}\) implies
	\(\thepair{b}{a} \isin \therelation{C}\),
	\(\thepair{a}{a} \isin \therelation{C}\), and
	\(\thepair{b}{b} \isin \therelation{C}\). However, symmetry and
	transitivity alone \emph{do not} imply that every \(x \in \theset{A}\)
	is represented by some pair in \(\therelation{C}\), which would indeed
	be required for reflexivity. That is, while it is true that the
	combination of symmetry and transitivity imply reflexivity among the
	subset of elements which do indeed have some other relation in
	\(\therelation{C}\), without reflexivity, it would still possible for
	some \(x\) to exist in \(\theset{A}\) such that there exists no \(y\) in
	\(\theset{A}\) such that \(\thepair{x}{y} \isin \therelation{C}\). For
	example, suppose \(\therelation{C}\) is a relation on \(\theset{A}\)
	such that \(\theset{A}\) is not the empty set but \(\therelation{C}\) is
	the empty set. This would be possible since the empty set is a subset of
	any other set, including
	\(\thecartesianproductof{\theset{A}}{\theset{A}}\). Both symmetry and
	transitivity would be vacuously true of \(\therelation{C}\) even though
	reflexivity would not hold.
\end{remark}

\end{document}
