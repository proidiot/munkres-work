\documentclass[main.tex]{subfiles}

\begin{document}

\subproblem{}\label{3a}
If \(x < 0\), then \(x^2 - x > 0\).

\begin{thm}[Implication holds in~\ref{3a}]
	Given \(x\) is a negative real number, then \(x^2 - x\) is positive.
	Symbolically, that is
	\[\forall x \in \mathbb{R}: x < 0 \implies x^2 - x > 0\]
\end{thm}
\begin{proof}
	Assume \(x\) is a negative real number. Since the multiplicative product
	of two negative real numbers is a positive real number, we know \(x^2\)
	is a positive real number. Also, since the additive inverse of a
	negative real number is its opposite positive real number, we can say
	that subtracting \(x\) is an equivalent operation to the addition of
	\(-x\) (which is a positive real number). Since we consequently know
	that \(x^2 - x\) is the sum of two positive real numbers, we know that
	\(x^2 - x\) is a positive real number (and, explicitly, \(x^2 - x > 0\))
	by the closure of the positive real numbers under addition.
\end{proof}

\begin{remark}[Contrapositive of~\ref{3a}]
	The statement
	\[x^2 - x \leq 0 \implies x \geq 0\]
	is the contrapositive of the statement
	\[x < 0 \implies x^2 - x > 0\]
\end{remark}
\begin{thm}[Implication holds for the contrapositive of~\ref{3a}]
	Given \(x\) is a real number such that \(x^2 - x\) is non-positive, then
	\(x\) is non-negative.
	Symbolically, that is
	\[\forall x \in \mathbb{R}: x^2 - x \leq 0 \implies x \geq 0\]
\end{thm}
\begin{proof}
	Assume \(x\) is a real number such that \(x^2 - x\) is non-positive. So
	\(x^2 - x\) is either negative or equal to zero.

	First, consider the case that \(x^2 - x\) is zero. By algebraic
	manipulation, we have that \(x^2\) is equal to \(x\). Since \(x\) is a
	real number, we know that this can only be true if \(x\) is either 0 or
	1. Since both 0 and 1 are non-negative, we know that \(x\) is
	non-negative.

	Second, consider the case that \(x^2 - x\) is negative. By algebraic
	manipulation, we have that \(x^2\) is less than \(x\). Since \(x\) is a
	real number, this could only be true if \(x\) is in the open interval
	between 0 and 1. Since this interval consists entirely of positive real
	numbers, we know that \(x\) must be a positive real number. By
	disjunctive introduction, we further know that \(x\) is non-negative.

	Therefore \(x^2 - x\) being non-positive implies that \(x\) is
	non-negative in all circumstances.
\end{proof}

\begin{remark}[Converse of~\ref{3a}]
	The statement
	\[x^2 - x > 0 \implies x < 0\]
	is the converse of the statement
	\[x < 0 \implies x^2 - x > 0\]
\end{remark}
\begin{thm}[Implication fails in the converse of~\ref{3a}]
	Given \(x\) is a real number such that \(x^2 - x\) is positive, it is
	not necessarilly the case that \(x\) is less than 0. Symbolically, that
	is
	\[\neg (\forall x \in \mathbb{R}: x^2 - x > 0 \implies x < 0)\]
\end{thm}
\begin{proof}
	Let \(x'\) be 2, which is in the real numbers. Now \(x'^2\) is 4, and
	\(x'^2 - x'\) is 2. Since 2 is greater than 0, we can choose \(x\) to be
	\(x'\) since doing so would cause \(x^2 - x\) to be greater than 0,
	which is our criteria. Note that \(x'\) is not less than 0. So having
	\(x\) be a real number such that \(x^2 - x\) is greater than 0 is
	insufficient for \(x\) to be less than 0.
\end{proof}

\end{document}
