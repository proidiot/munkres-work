\documentclass[main.tex]{subfiles}

\begin{document}

\subproblem{}\label{s2p4d}

If \(f\) and \(g\) are surjective, show that \(\compose{g}{f}\) is surjective.

\begin{thm}
	The composition of surjective functions is surjective. So, given
	\(\mapping{f}{A}{B}\) and \(\mapping{g}{B}{C}\) are each surjective,
	then \(\compose{g}{f}\) must also be surjective.
\end{thm}
\begin{proof}
	Let \(A\), \(B\), and \(C\) be sets, let \(f\) be a surjective mapping
	from \(A\) to \(B\), and let \(g\) be a surjective mapping from \(B\) to
	\(C\).

	Assume \(z\) is in \(C\). Since \(g\) is surjective, we know there
	exists \(y\) in \(B\) such that \(z = g(y)\). Since \(f\) is surjective,
	we know there exists \(x\) in \(A\) such that \(y = f(x)\). Recalling
	that \(z = g(y)\), we have \(z = g(f(x))\) by substitution. By the
	definition of function composition, we can say
	\(z = (\compose{g}{f})(x)\). Since we have shown an arbitrary choice in
	the image set of \(\compose{g}{f}\) implies the existence of a
	corresponding element in the domain of \(\compose{g}{f}\), we have
	demonstrated the surjectivity of \(\compose{g}{f}\).
\end{proof}

\end{document}
