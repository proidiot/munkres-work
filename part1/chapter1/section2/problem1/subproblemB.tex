\documentclass[main.tex]{subfiles}

\begin{document}

\subproblem{}\label{s2p1b}

Show that \(f(\inverse{f}(B_0)) \subset B_0\) and that equality holds if \(f\)
is surjective.

\begin{thm}\thlabel{s2p1bt1}
	Given a function \(f\) which maps \(A\) to \(B\), and given a subset
	\(B_0\) of \(B\), then the image of the preimage of \(B_0\) under \(f\)
	is a subset of \(B_0\). Symbolically, that is:
	\[f(\inverse{f}(B_0)) \subset B_0\]
\end{thm}
\begin{proof}
	Let \(A\) and \(B\) be sets, let \(f\) be a function mapping \(A\) onto
	\(B\), and let \(B_0\) be a subset of \(B\). For brevity, we'll refer to
	\(\inverse{f}(B_0)\) as \(A'\) and \(f(\inverse{f}(B_0))\) as \(B'\) (so
	\(A'\) is the preimage of \(B_0\) under \(f\) and \(B'\) is the image of
	\(A'\) under \(f\)). Assume \(x\) is some arbitrary element of \(B'\).
	By the definition of image, we know that \(B'\) is the set of all values
	\(b\) such that \(b = f(a)\) for some value \(a\) in \(A'\). Since \(x\)
	is in \(B'\), that must mean there is some value \(y\) in \(A'\) such
	that \(x = f(y)\). Further, since \(y\) is in \(A'\), we know that there
	exists some \(x'\) in \(B_0\) such that \(x' = f(y)\). As we have
	already established that \(x = f(y)\), the rule of assignment gives us
	that \(x = x'\). So \(x\) is in \(B_0\), and thus
	\(f(\inverse{f}(B_0))\) is a subset of \(B_0\).
\end{proof}

\begin{thm}\thlabel{s2p1bt2}
	Given a function \(f\) which maps \(A\) to \(B\), and given a subset
	\(B_0\) of \(B\), and given the surjectivity of \(f\), then the image of
	the preimage of \(B_0\) under \(f\) is a subset of \(B_0\).
	Symbolically, that is:
	\[f(\inverse{f}(B_0)) = B_0\]
\end{thm}
\begin{proof}
	Let \(A\) and \(B\) be sets, let \(f\) be a function mapping \(A\) onto
	\(B\), let \(B_0\) be a subset of \(B\), and let \(f\) be surjective.
	Note that \(f(\inverse{f}(B_0))\) is a subset of \(B_0\) since the
	assumptions of~\thref{s2p1bt1} are a weaker subset of the current
	assumptions.

	For brevity, we'll refer to \(\inverse{f}(B_0)\) as \(A'\) and
	\(f(\inverse{f}(B_0))\) as \(B'\) (so \(A'\) is the preimage of \(B_0\)
	under \(f\) and \(B'\) is the image of \(A'\) under \(f\)). Assume \(x\)
	is an arbitrary member of \(B_0\). By the definition of surjectivity, we
	have that \(x = f(a)\) for at least one value of \(a\) in \(A\); we'll
	call the particular value \(y\) (i.e., \(x = f(y)\) and \(y\) is in
	\(A\)). By the definition of preimage, \(y\) is in \(A'\) since \(x\) is
	in \(B_0\). Now considering the value \(x' = f(y)\), the definition of
	image gives us that \(x'\) is in \(B'\). Since \(x = f(y)\), the rule of
	assignment gives us \(x = x'\). So \(x\) is in \(B'\), or more
	explicitly, \(x\) is in \(f(\inverse{f}(B_0))\). Thus \(B_0\) is a
	subset of \(f(\inverse{f}(B_0))\).

	Since both \(f(\inverse{f}(B_0))\) is a subset of \(B_0\) and \(B_0\) is
	a subset of \(f(\inverse{f}(B_0))\), we have established equality.
\end{proof}


\end{document}
