\documentclass[main.tex]{subfiles}

\begin{document}

\subproblem{}\label{9a}

\begin{thm}[De Morgan's First Law]
	Set difference across arbitrary set union is equivalent to arbitrary set
	intersection of set differences. Symbolically, given a set \(A\) and a
	collection of sets \(\metaset{B}\):
	\[A \without \Arbitraryunion{B \in \metaset{B}}{B} = \Arbitraryintersection{B \in \metaset{B}}{A \without B}\]
\end{thm}
\begin{proof}
	We will establish set equality by demonstrating that each set it a
	subset of the other. Let \(A\) be a set and \(\metaset{B}\) be a
	collection of sets.

	\medskip{}
	First we'll consider \(A \without \arbitraryunion{B \in \metaset{B}}{B}\).
	In the trivial case that the set is empty, then it is a subset of
	\(\arbitraryintersection{B \in \metaset{B}}{A \without B}\) since the
	empty set is a subset of all sets. Instead, let \(x\) be an arbitrary
	value in \(A \without \arbitraryunion{B \in \metaset{B}}{B}\). By the
	definition of set difference, we know that \(x\) is in \(A\) and \(x\)
	is not in \(\arbitraryunion{B \in \metaset{B}}{B}\). Recall that the
	converse of~\ref{5a} (i.e.,~\thref{5atc}) is true. Since the
	contrapositive of a statement has the same truth value as the original
	statement, we know that the contrapositive of the converse of~\ref{5a}
	(i.e.,~\ref{6b}) is also true. So we have that \(x\) not being in
	\(\arbitraryunion{B \in \metaset{B}}{B}\) implies that \(x\) is in
	\(\setbuild{x'}{\forall B \in \metaset{B}, x' \notin B}\). As we have
	already established that \(x\) is not in
	\(\arbitraryunion{B \in \metaset{B}}{B}\), modus ponens shows that for
	all \(B\) in \(\metaset{B}\), we know that \(x\) is not in \(B\). Now
	since we established previously that \(x\) is indeed in \(A\), we can
	see that for all \(B\) in \(\metaset{B}\), we know simultaneously that
	\(x\) is in \(A\) and \(x\) is not in \(B\). By the definition of set
	difference, for all \(B\) in \(\metaset{B}\), we see that \(x\) is in
	\(A \without B\). By the definition of arbitrary intersection, we have
	that \(x\) is in
	\(\arbitraryintersection{B \in \metaset{B}}{A \without B}\). So
	\(A \without \arbitraryunion{B \in \metaset{B}}{B}\) is a subset of
	\(\arbitraryintersection{B \in \metaset{B}}{A \without B}\).

	\medskip{}
	Second, we'll consider
	\(\arbitraryintersection{B \in \metaset{B}}{A \without B}\). In the
	trivial case that the set is empty, then it is a subset of
	\(A \without \arbitraryunion{B \in \metaset{B}}{B}\) since the empty set
	is a subset of all sets. Instead, let \(x\) be an arbitrary value in
	\(\arbitraryintersection{B \in \metaset{B}}{A \without B}\). By the
	definition of arbitrary set intersection, we know that \(x\) is in
	\(A \without B\) for every \(B\) in \(\metaset{B}\). By the definition
	of set difference, we know that \(x\) is in \(A\) and \(x\) is not in
	\(B\) for every \(B\) in \(\metaset{B}\). So we have that \(x\) is in
	\(A\), and separately \(x\) is not in \(B\) for every \(B\) in
	\(\metaset{B}\). Since~\ref{5a} (i.e.,~\thref{5at}) is true, we know
	that~\ref{6a} is also true since a statement has the same truth value as
	its contrapositive. Combined with the fact that we have already
	established that for all \(B\) in \(\metaset{B}\) that \(x\) is not in
	\(B\), modus ponens gives us that \(x\) is not in
	\(\arbitraryunion{B \in \metaset{B}}{B}\). Since we previously
	established that \(x\) is in \(A\), the definition of set difference
	gives us \(x\) being in
	\(A \without \arbitraryunion{B \in \metaset{B}}{B}\). So
	\(\arbitraryintersection{B \in \metaset{B}}{A \without B}\) is a subset
	of \(A \without \arbitraryunion{B \in \metaset{B}}{B}\).

	\medskip{}
	As we have demonstrated that
	\(A \without \arbitraryunion{B \in \metaset{B}}{B}\) and
	\(\arbitraryintersection{B \in \metaset{B}}{A \without B}\) are subsets
	of each other, we have also demonstrated equality among these sets.
\end{proof}

\end{document}
