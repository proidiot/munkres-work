\documentclass[main.tex]{subfiles}

\begin{document}

\subproblem{}\label{s2p5e}

Show that if \(f\) has both a left inverse \(g\) and a right inverse \(h\), then
\(f\) is bijective and \(g = h = \inverse{f}\).

\begin{thm}
	A function with both a left and a right inverse is bijective.
\end{thm}
\begin{proof}
	Let \(f\) be a function with both a left inverse and a right inverse. By
	\thref{s2p5at3}, we know \(f\) is injective. By \thref{s2p5at4}, we know
	\(f\) is surjective. Therefore by definition, we know that \(f\) is
	bijective.
\end{proof}

\begin{thm}
	If a function has both a left and a right inverse, then these inverses
	are equal and the inverse is unique.
\end{thm}
\begin{proof}
	Let \(A\) and \(B\) be sets, and let \(\mapping{f}{A}{B}\) be a function
	with both a left inverse \(g\) and a right inverse \(h\). By
	\thref{s2p5at3} we know \(f\) is injective, and so by \thref{s2p5dt4} we
	know \(h\) is the distinct right inverse. By \thref{s2p5at4} we know
	\(f\) is surjective, and so by \thref{s2p5dt2} we know \(g\) is the
	distinct left inverse.

	Assume \(y\) is in \(B\). Since \(f\) is surjective, we know there
	exists \(x\) in \(A\) such that \(y = f(x)\). Recalling that the
	definition of right inverse gives that \(f(h(y)) = y\), we know
	\(h(y) = x\) by the injectivity of \(f\). Also, recalling that the
	definition of left inverse gives that \(g(f(x)) = x\), we have that
	\(g(y) = x\) by substitution. So we have that \(g(y) = x = h(y)\) for
	arbitrary \(y\) in \(B\). Thus \(g = h\). Therefore
	\(g = h = \inverse{f}\) is the unique inverse of \(f\).
\end{proof}

\end{document}
