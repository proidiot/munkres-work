\documentclass[main.tex]{subfiles}

\begin{document}

\subproblem{}\label{2k}
The converse of~\ref{2j}.
\begin{remark}[The converse of~\ref{2j}]
	The statement
	\[(A \times B) \subset (C \times D) \implies A \subset C \land B \subset D\]
	is the converse of
	\[A \subset C \land B \subset D \implies (A \times B) \subset (C \times D)\]
\end{remark}

\begin{thm}[Implication fails in~\ref{2k}]
	Given sets \(A\), \(B\), \(C\), and \(D\), it is not the case that the
	cartesian product of \(A\) and \(B\) being a subset of the cartesian
	product of \(C\) and \(D\) implies that both \(A\) is a subset of \(C\)
	and \(B\) is a subset of \(D\). Symbolically, that is
	\[\neg [(A \times B) \subset (C \times D) \implies A \subset C \land B \subset D]\]
\end{thm}
\begin{proof}
	Choose \(A\) to be the empty set, \(B\) to be the odds, \(C\)
	to be the integers, and \(D\) to be the evens. Since \(A\) is the empty
	set, the cartesian product of \(A\) and \(B\) is also the empty set.
	Since the empty set is a subset of all sets, \(A \times B\) is a subset
	of \(C \times D\). Now suppose it was the case that \(A \times B\) being
	a subset of \(C \times D\) really did imply that both \(A\) was a subset
	of \(C\) and \(B\) was a subset of \(D\). Given our particular choices,
	that would mean (among other things) that the set of all odds is a
	subset of the set of all evens, which is absurd. So given our choices it
	is not the case that \(A \times B\) being a subset of \(C \times D\)
	implies that both \(A\) is a subset of \(C\) and \(B\) is a subset of
	\(D\).
\end{proof}

\end{document}

