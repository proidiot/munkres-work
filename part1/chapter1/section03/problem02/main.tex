\makeatletter
\def\input@path{{../}}
\makeatother
\documentclass[../main.tex]{subfiles}

\begin{document}

\problem{}\label{s03p02}

Let \(C\) be a relation on a set \(A\). If \(A_0 \isasubsetof A\), define the
\keyword{restriction} of \(C\) to \(A_0\) to be the relation
\(C \intersect (A_0 \cross A_0)\). Show that the restriction of an equivalence
relation is an equivalence relation.

\begin{thm}\thlabel{s03p02t1}
	Given \(\therelation{C}\) is a relation on \(\theset{A}\) and
	\(\theset{A_0}\) is a subset of \(\theset{A}\), the restriction of
	\(\therelation{C}\) to \(\theset{A_0}\) is a relation on
	\(\theset{A_0}\).
\end{thm}
\begin{proof}
	Let \(\therelation{C}\) be a relation on \(\theset{A}\). So
	\(\therelation{C}\) is a subset of
	\(\thecartesianproductof{\theset{A}}{\theset{A}}\). Let \(\theset{A_0}\)
	be a subset of \(\theset{A}\). By the definition of the restriction of a
	relation, the restriction of \(\therelation{C}\) to \(\theset{A_0}\) is
	a subset of
	\(\therelation{C} \intersect \thecartesianproductof{\theset{A_0}}{\theset{A_0}}\).
	Let \(\thepair{x}{y}\) be in the restriction of \(\therelation{C}\) to
	\(\theset{A_0}\). So \(\thepair{x}{y} \isin \theintersectionof{\therelation{C}}{\thecartesianproductof{\theset{A_0}}{\theset{A_0}}}\).
	By the definition of set intersection,
	\(\thepair{x}{y} \isin \therelation{C}\) and
	\(\thepair{x}{y} \isin \thecartesianproductof{\theset{A_0}}{\theset{A_0}}\).
	Since \(\thepair{x}{y}\) being in the restriction of \(\therelation{C}\)
	to \(\theset{A_0}\) implies that \(\thepair{x}{y}\) is also in
	\(\thecartesianproductof{\theset{A_0}}{\theset{A_0}}\), we have shown
	that the restriction of \(\therelation{C}\) to \(\theset{A_0}\) is a
	subset of \(\thecartesianproductof{\theset{A_0}}{\theset{A_0}}\). Thus,
	be the definition of a relation, the restriction of \(\therelation{C}\)
	to \(\theset{A_0}\) is a relation on \(\theset{A_0}\).
\end{proof}

\begin{thm}\thlabel{s03p02t2}
	Given \(\therelation{C}\) is a reflexive relation on \(\theset{A}\) and
	\(\theset{A_0}\) is a subset of \(\theset{A}\), the restriction of
	\(\therelation{C}\) to \(\theset{A_0}\) is a reflexive relation on
	\(\theset{A_0}\).
\end{thm}
\begin{proof}
	Let \(\therelation{C}\) be a reflexive relation on \(\theset{A}\). Let
	\(\theset{A_0}\) be a subset of \(\theset{A}\). By \thref{s03p02t1},
	the restriction of \(\therelation{C}\) to \(\theset{A_0}\) is a relation
	on \(\theset{A_0}\). Let \(x\) be some element of \(\theset{A_0}\).
	Since \(\theset{A_0}\) is a subset of \(\theset{A}\), we know that
	\(x \isin \theset{A}\) as well. Since \(\therelation{C}\) is reflexive,
	we know that \(\thepair{x}{x}\) must be in \(\therelation{C}\). Note
	that by the definition of the cartesian product, \(x \in \theset{A_0}\)
	implies that
	\(\thepair{x}{x} \isin \thecartesianproductof{\theset{A_0}}{\theset{A_0}}\).
	Since \(\thepair{x}{x}\) is in both \(\therelation{C}\) and
	\(\thecartesianproductof{\theset{A_0}}{\theset{A_0}}\), the definition
	of set intersection gives us that
	\(\thepair{x}{x} \isin \theintersectionof{\therelation{C}}{\thecartesianproductof{\theset{A_0}}{\theset{A_0}}}\).
	By the definition of the restriction of a relation, we see that
	\(\thepair{x}{x}\) is in the restriction of \(\therelation{C}\) to
	\(\theset{A_0}\). Since \(x \in \theset{A_0}\) implies
	\(\thepair{x}{x}\) is in the restriction of \(\therelation{C}\) to
	\(\theset{A_0}\), we have thus shown that the restriction of
	\(\therelation{C}\) to \(\theset{A_0}\) is a reflexive relation.
\end{proof}

\begin{thm}\thlabel{s03p02t3}
	Given \(\therelation{C}\) is a symmetric relation on \(\theset{A}\) and
	\(\theset{A_0}\) is a subset of \(\theset{A}\), the restriction of
	\(\therelation{C}\) to \(\theset{A_0}\) is a symmetric relation on
	\(\theset{A_0}\).
\end{thm}
\begin{proof}
	Let \(\therelation{C}\) be a symmetric relation on \(\theset{A}\). Let
	\(\theset{A_0}\) be a subset of \(\theset{A}\). By \thref{s03p02t1},
	the restriction of \(\therelation{C}\) to \(\theset{A_0}\) is a relation
	on \(\theset{A_0}\). Let \(\thepair{x}{y}\) be in the restriction of
	\(\therelation{C}\) to \(\theset{A_0}\). So \(x \isin \theset{A_0}\) and
	\(y \isin \theset{A_0}\). Note that by the definition of
	the cartesian product,
	\(\thepair{y}{x} \isin \thecartesianproductof{\theset{A_0}}{\theset{A_0}}\).
	By the definition of the restriction of a relation, \(\thepair{x}{y}\)
	being in the restriction of \(\therelation{C}\) to \(\theset{A_0}\)
	implies
	\(\thepair{x}{y} \isin \theintersectionof{\therelation{C}}{\thecartesianproductof{\theset{A_0}}{\theset{A_0}}}\),
	and so further \(\thepair{x}{y} \isin \therelation{C}\). Now since
	\(\therelation{C}\) is symmetric, we know that
	\(\thepair{y}{x} \isin \therelation{C}\) as well. Since
	\(\thepair{y}{x}\) is in both \(\therelation{C}\) and
	\(\thecartesianproductof{\theset{A_0}}{\theset{A_0}}\), the definition
	of set intersection gives us that
	\(\thepair{y}{x} \isin \theintersectionof{\therelation{C}}{\thecartesianproductof{\theset{A_0}}{\theset{A_0}}}\).
	By the definition of the restriction of a relation, we see that
	\(\thepair{y}{x}\) is in the restriction of \(\therelation{C}\) to
	\(\theset{A_0}\). Since \(\thepair{x}{y}\) being in the restriction of
	\(\therelation{C}\) to \(\theset{A_0}\) is sufficient to show that
	\(\thepair{y}{x}\) is also in the restriction of \(\therelation{C}\) to
	\(\theset{A_0}\), we have thus shown that the restriction of
	\(\therelation{C}\) to \(\theset{A_0}\) is a symmetric relation.
\end{proof}

\begin{thm}\thlabel{s03p02t4}
	Given \(\therelation{C}\) is a transitive relation on \(\theset{A}\) and
	\(\theset{A_0}\) is a subset of \(\theset{A}\), the restriction of
	\(\therelation{C}\) to \(\theset{A_0}\) is a transitive relation on
	\(\theset{A_0}\).
\end{thm}
\begin{proof}
	Let \(\therelation{C}\) be a transitive relation on \(\theset{A}\). Let
	\(\theset{A_0}\) be a subset of \(\theset{A}\). By \thref{s03p02t1},
	the restriction of \(\therelation{C}\) to \(\theset{A_0}\) is a relation
	on \(\theset{A_0}\). Let \(\thepair{x}{y}\) and \(\thepair{y}{z}\) be in
	the restriction of \(\therelation{C}\) to \(\theset{A_0}\). So
	\(x \isin \theset{A_0}\), \(y \isin \theset{A_0}\), and
	\(z \isin \theset{A_0}\). Note that by the definition of the cartesian
	product,
	\(\thepair{x}{z} \isin \thecartesianproductof{\theset{A_0}}{\theset{A_0}}\).
	By the definition of the restriction of a relation, \(\thepair{x}{y}\)
	and \(\thepair{y}{z}\) being in the restriction of \(\therelation{C}\)
	to \(\theset{A_0}\) implies both
	\(\thepair{x}{y} \isin \theintersectionof{\therelation{C}}{\thecartesianproductof{\theset{A_0}}{\theset{A_0}}}\)
	and
	\(\thepair{y}{z} \isin \theintersectionof{\therelation{C}}{\thecartesianproductof{\theset{A_0}}{\theset{A_0}}}\),
	and so further both \(\thepair{x}{y} \isin \therelation{C}\) and
	\(\thepair{y}{z} \isin \therelation{C}\). Now since \(\therelation{C}\)
	is transitive, we know that \(\thepair{x}{z} \isin \therelation{C}\) as
	well. Since \(\thepair{x}{z}\) is in both \(\therelation{C}\) and
	\(\thecartesianproductof{\theset{A_0}}{\theset{A_0}}\), the definition
	of set intersection gives us that
	\(\thepair{x}{z} \isin \theintersectionof{\therelation{C}}{\thecartesianproductof{\theset{A_0}}{\theset{A_0}}}\).
	By the definition of the restriction of a relation, we see that
	\(\thepair{x}{z}\) is in the restriction of \(\therelation{C}\) to
	\(\theset{A_0}\). Since \(\thepair{x}{y}\) and \(\thepair{y}{z}\) being
	in the restriction of \(\therelation{C}\) to \(\theset{A_0}\) is
	sufficient to show that \(\thepair{x}{z}\) is also in the restriction of
	\(\therelation{C}\) to \(\theset{A_0}\), we have thus shown that the
	restriction of \(\therelation{C}\) to \(\theset{A_0}\) is a transitive
	relation.
\end{proof}

\begin{thm}\thlabel{s03p02t5}
	Given \(\therelation{C}\) is an equivalence relation on \(\theset{A}\)
	and \(\theset{A_0}\) is a subset of \(\theset{A}\), the restriction of
	\(\therelation{C}\) to \(\theset{A_0}\) is an equivalence relation on
	\(\theset{A_0}\).
\end{thm}
\begin{proof}
	Let \(\therelation{C}\) be an equivalence relation on \(\theset{A}\).
	By the definition of an equivalence relation, \(\therelation{C}\) is
	reflexive, symmetric, and transitive. Let \(\theset{A_0}\) be a subset
	of \(\theset{A}\). By \thref{s03p02t1}, the restriction of
	\(\therelation{C}\) to \(\theset{A_0}\) is a relation on
	\(\theset{A_0}\). By \thref{s03p02t2}, the restriction of
	\(\therelation{C}\) to \(\theset{A_0}\) is reflexive. By
	\thref{s03p02t3}, the restrction of \(\therelation{C}\) to
	\(\theset{A_0}\) is symmetric. By \thref{s03p02t4}, the restriction of
	\(\therelation{C}\) to \(\theset{A_0}\) is transitive. Therefore the
	restriction of \(\therelation{C}\) to \(\theset{A_0}\) is an equivalence
	relation.
\end{proof}

\end{document}
