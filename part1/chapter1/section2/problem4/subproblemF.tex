\documentclass[main.tex]{subfiles}

\begin{document}

\subproblem{}\label{s2p4f}

Summarize your answers to \(\ref{s2p4b} \thru~\ref{s2p4e}\) in the form of a
theorem.

\begin{thm}\thlabel{s2p4ft1}
	Given functions \(\mapping{f}{A}{B}\) and \(\mapping{g}{B}{C}\), if
	\(f\) is injective and the restrictuon of \(g\) to the image of \(A\)
	under \(f\) is also injective, then \(\compose{g}{f}\) is injective.
\end{thm}
\begin{proof}
	\todo{}
\end{proof}

\begin{thm}
	Given functions \(\mapping{f}{A}{B}\) and \(\mapping{g}{B}{C}\), we can
	say that \(\compose{g}{f}\) is injective iff \(f\) is injective and the
	restrictuon of \(g\) to the image of \(A\) under \(f\) is also
	injective.
\end{thm}
\begin{proof}
	Let \(A\), \(B\), and \(C\) be sets, let \(f\) be a mapping from \(A\)
	to \(B\), and let \(g\) be a mapping from \(B\) to \(C\).

	First, assume \(\compose{g}{f}\) is injective. By \thref{s2p4ct1}, we
	have that \(f\) is injective. Also, by \thref{s2p4ct3}, we have that
	the restriction of \(g\) to the image of \(A\) under \(f\) is injective.
	Thus by conjunction, we have that \(f\) is injective and
	\(\restrict{g}{f(A)}\) is injective.

	Next, assume \(f\) is injective and \(\restrict{g}{f(A)}\) is injective.
	Thus by \thref{s2p4ft1}, we have that \(\compose{g}{f}\) is injective.

	As we have demonstrated that each implies the other, we have therefore
	demonstrated that \(\compose{g}{f}\) is injective iff \(f\) is injective
	and \(\restrict{g}{f(A)}\) is injective.
\end{proof}

\begin{thm}
	Given functions \(\mapping{f}{A}{B}\) and \(\mapping{g}{B}{C}\), we can
	say that \(\compose{g}{f}\) is surjective iff the restrictuon of \(g\)
	to the image of \(A\) under \(f\) is surjective.
\end{thm}
\begin{proof}
	\todo{}
\end{proof}

\end{document}
