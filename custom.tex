\renewcommand{\qedsymbol}{$\blacksquare$}
\renewcommand{\thesection}{\S\arabic{section}}
\renewcommand{\thesubsection}{Problem \arabic{subsection}}
\renewcommand{\thesubsubsection}{\thesubsection(\alph{subsubsection})}

\newtheorem{thm}{Theorem}
\newtheorem*{remark}{Remark}

\counterwithout*{chapter}{part}
\counterwithout*{section}{chapter}
\setcounter{secnumdepth}{3}
\setcounter{tocdepth}{2}

% from https://tex.stackexchange.com/questions/194798/change-vertical-space-in-overset
% new \oset macro
\makeatletter
\newcommand{\oset}[3][0ex]{%
  \mathrel{\mathop{#3}\limits^{
    \vbox to#1{\kern-2\ex@
    \hbox{$\scriptstyle#2$}\vss}}}}
\makeatother

\newcommand{\card}[1] {\vert{} #1 \vert{}}
\newcommand{\sunion}{\cup}
\newcommand{\union}{\cup}
\newcommand{\sintersect}{\cap}
\newcommand{\intersect}{\cap}
\newcommand{\sdiff}{\setminus}
\newcommand{\powerset}[1]{\mathcal{P}\left(#1\right)}
\newcommand{\metaset}[1]{\mathscr{#1}}
\newcommand{\setbuild}[2]{\{#1|#2\}}
\newcommand{\suchthat}{:}
\newcommand{\cross}{\times}
\newcommand{\without}{\setminus}
\newcommand{\isasubsetof}{\subset}
\newcommand{\isasupersetof}{\supset}
\newcommand{\generalizedunion}[2]{\bigcup_{#1} {#2}}
\newcommand{\arbitraryunion}[2]{\bigcup_{#1} {#2}}
\newcommand{\Arbitraryunion}[2]{\bigcup\limits_{#1} {#2}}
\newcommand{\generalizedintersection}[2]{\bigcap_{#1} {#2}}
\newcommand{\arbitraryintersection}[2]{\bigcap_{#1} {#2}}
\newcommand{\Arbitraryintersection}[2]{\bigcap\limits_{#1} {#2}}
\newcommand{\isin}{\in}
\newcommand{\thereexists}{\exists}
\newcommand{\itisnothecasethat}[1]{\neg{} (#1)}
\newcommand{\naturals}{\mathbb{N}}
\newcommand{\integers}{\mathbb{Z}}
\newcommand{\reals}{\mathbb{R}}
\newcommand{\problem}{\subsection{}}
\newcommand{\subproblem}{\subsubsection{}}
\newcommand{\mapping}[3]{{#1}: {#2} \mapsto{} {#3}}
\newcommand{\inverse}[1]{{#1}^{-1}}
\newcommand{\imageunder}[2]{{#1}({#2})}
\newcommand{\preimageunder}[2]{\inverse{#1}({#2})}
\newcommand{\compose}[2]{{#1} \circ{} {#2}}
\newcommand{\composeapply}[3]{(\compose{#1}{#2})({#3})}
\newcommand{\thru}{-}
\newcommand{\definedterm}[1]{\emph{\textbf{#1}}}
\newcommand{\ceil}[1]{\left \lceil #1 \right \rceil}
\newcommand{\restrict}[2]{{#1}\restriction_{#2}}
\newcommand{\true}{\mbox{TRUE}}
\newcommand{\false}{\mbox{FALSE}}
\newcommand{\booleans}{\{\true,\false\}}
\newcommand{\eqmod}[3]{{#1} \equiv {#2} \mod {#3}}
\newcommand{\maybeeqmod}[3]{{#1} \oset{?}{\equiv} {#2} \mod {#3}}
\newcommand{\plusorminus}{\pm}
\newcommand{\posintegers}{{\integers}_{+}}
\newcommand{\openinterval}[2]{({#1},{#2})}
\newcommand{\closedinterval}[2]{[{#1},{#2}]}
\newcommand{\lopenrclosedinterval}[2]{({#1},{#2}]}
\newcommand{\lclosedropeninterval}[2]{[{#1},{#2})}

