\documentclass[main.tex]{subfiles}

\begin{document}

\subproblem{}\label{s2p4e}

If \(\compose{g}{f}\) is surjective, what can you say about surjectivity of
\(f\) and \(g\)?

\begin{thm}
	Given functions \(\mapping{f}{A}{B}\) and \(\mapping{g}{B}{C}\), if
	\(\compose{g}{f}\) is surjective, then the restriction of \(g\) to the
	image of \(A\) under \(f\) (i.e. \(\restrict{g}{f(A)}\)) must also be
	surjective.
\end{thm}
\begin{proof}
	Let \(A\), \(B\), and \(C\) be sets. Let \(f\) be a mapping from \(A\)
	to \(B\) and let \(g\) be a mapping from \(B\) to \(C\) such that
	\(\compose{g}{f}\) is surjective.

	Assume \(z\) is in \(C\). Since \(\compose{g}{f}\) is surjective, we
	know there exists \(x\) in \(A\) such that \(z = (\compose{g}{f})(x)\).
	By the definition of function composition, we know \(z = g(f(x))\).
	Since \(f(x)\) is in the image of \(A\) under \(f\), we know
	\(\restrict{g}{f(A)}\) is also surjective.
\end{proof}

\begin{thm}
	Given functions \(\mapping{f}{A}{B}\) and \(\mapping{g}{B}{C}\), if
	\(\compose{g}{f}\) is surjective, then \(f\) is not necessarilly also
	surjective.
\end{thm}
\begin{proof}
	\todo{}
\end{proof}

\end{document}
