\makeatletter
\def\input@path{{../}}
\makeatother
\documentclass[../main.tex]{subfiles}

\begin{document}

\problem{}\label{s03p01}

Define two points \((x_0,y_0)\) and \((x_1,y_1)\) of the plane to be equivalent
if \(y_0 - {x_0}^2 = y_1 - {x_1}^2\). Check that this is an equivalence relation
and describe the equivalence classes.

\begin{thm}\thlabel{s03p01t1}
	The relation \(\thicksim\) on \({\reals}^2\) such that
	\(({x_0},{y_0})\thicksim({x_1},{y_1})\) if and only if
	\({y_0}-{{x_0}^2}={y_1}-{{x_1}^2}\) is reflexive.
\end{thm}
\begin{proof}
	Let \(\thicksim\) be a relation on \({\reals}^2\) such that
	\(({x_0},{y_0})\thicksim({x_1},{y_1})\) if and only if
	\({y_0}-{{x_0}^2}={y_1}-{{x_1}^2}\). Now consider some arbitrary
	element \((a,b) \in {\reals}^2\). Note that \(a\) and \(b\) are members
	of \(\reals\). By the closure of \(\reals\) under multiplication, we
	have that \({a}^2 \isin \reals\), and by the closure of \(\reals\) under
	subtraction, we have that \(b-{{a}^2} \isin \reals\). By the reflexivity
	of the equivalence relation \(=\) of \(\reals\), we have that
	\(b-{{a}^2} = b-{{a}^2}\). Thus \((a,b)\thicksim(a,b)\) by the
	definition of \(\thicksim\). Therefore \(\thicksim\) is reflexive.
\end{proof}

\begin{thm}\thlabel{s03p01t2}
	The relation \(\thicksim\) on \({\reals}^2\) such that
	\(({x_0},{y_0})\thicksim({x_1},{y_1})\) if and only if
	\({y_0}-{{x_0}^2}={y_1}-{{x_1}^2}\) is symmetric.
\end{thm}
\begin{proof}
	Let \(\thicksim\) be a relation on \({\reals}^2\) such that
	\(({x_0},{y_0}) \thicksim ({x_1},{y_1})\) if and only if
	\({y_0} - {{x_0}^2} = {y_1} - {{x_1}^2}\). Now consider elements
	\((a,b)\) and \((c,d)\) in \({\reals}^2\) such that
	\((a,b) \thicksim (c,d)\). By the definition of \(\thicksim\), we know
	that \(b - {{a}^2} = d - {{c}^2}\). Note that each of \(a\), \(b\),
	\(c\), and \(d\) are members of \(\reals\). By the closure of \(\reals\)
	under multiplication, we have that \({a}^2 \isin \reals\) and
	\({c}^2 \isin \reals\). By the closure of \(\reals\) under subtraction,
	we have that \(b-{{a}^2} \isin \reals\) and \(d-{{c}^2} \isin \reals\).
	Since we have already shown that \(b - {{a}^2} = d - {{c}^2}\), the
	symmetry of the equivalence relation \(=\) of \(\reals\) gives us that
	\(d-{{c}^2} = b-{{a}^2}\). Thus \((c,d)\thicksim(a,b)\) by the
	definition of \(\thicksim\). Therefore \(\thicksim\) is symmetric.
\end{proof}

\begin{thm}\thlabel{s03p01t3}
	The relation \(\thicksim\) on \({\reals}^2\) such that
	\(({x_0},{y_0})\thicksim({x_1},{y_1})\) if and only if
	\({y_0}-{{x_0}^2}={y_1}-{{x_1}^2}\) is transative.
\end{thm}
\begin{proof}
	Let \(\thicksim\) be a relation on \({\reals}^2\) such that
	\(({x_0},{y_0}) \thicksim ({x_1},{y_1})\) if and only if
	\({y_0} - {{x_0}^2} = {y_1} - {{x_1}^2}\). Now consider elements
	\((a,b)\), \((c,d)\), and \((e,f)\) in \({\reals}^2\) such that
	\((a,b) \thicksim (c,d)\) and \((c,d) \thicksim (e,f)\). By the
	definition of \(\thicksim\), we know that \(b - {{a}^2} = d - {{c}^2}\)
	and \(d - {{c}^2} = f - {{e}^2}\). Note that each of \(a\), \(b\),
	\(c\), \(d\), \(e\), and \(f\) are members of \(\reals\). By the closure
	of \(\reals\) under multiplication, we have that \({a}^2 \isin \reals\),
	\({c}^2 \isin \reals\), and \({e}^2 \isin \reals\). By the closure of
	\(\reals\) under subtraction, we have that \(b - {{a}^2} \isin \reals\),
	\(d - {{c}^2} \isin \reals\), and \(f - {{e}^2} \isin \reals\). Since we
	have already shown that \(b - {{a}^2} = d - {{c}^2}\) and
	\(d - {{c}^2} = f - {{e}^2}\), the transitivity of the equivalence
	relation \(=\) of \(\reals\) gives us that
	\(b - {{a}^2} = f - {{e}^2}\). Thus \((a,b)\thicksim(e,f)\) by the
	definition of \(\thicksim\). Therefore \(\thicksim\) is transitive.
\end{proof}

\begin{thm}\thlabel{s03p01t4}
	The relation \(\thicksim\) on \({\reals}^2\) such that
	\(({x_0},{y_0})\thicksim({x_1},{y_1})\) if and only if
	\({y_0}-{{x_0}^2}={y_1}-{{x_1}^2}\) is an equivalence relation.
\end{thm}
\begin{proof}
	Let \(\thicksim\) be a relation on \({\reals}^2\) such that
	\(({x_0},{y_0}) \thicksim ({x_1},{y_1})\) if and only if
	\({y_0} - {{x_0}^2} = {y_1} - {{x_1}^2}\). Since \(\thicksim\) is
	reflexive (by \thref{s03p01t1}), symmetric (by \thref{s03p01t2}), and
	transitive (by \thref{s03p01t3}), we have that \(\thicksim\) is, by
	definition, an equivalence relation.
\end{proof}

In each of the preceding proofs, we established the desired property can be
shown from the already-established equivalence relation \(=\) of \(\reals\). In
order to do so, we had to demonstrate, for some point \((a,b) \in {\reals}^2\),
that \(b - {{a}^2}\) was in \(\reals\) and equal to some other specific value.
But what value would this be? For the moment, consider the point
\((0,0) \in {\reals}^2\). By simplification we have that \(0 - {{0}^2} = 0\). We
can see that any other point \((c,d) \in {\reals}^2\) such that
\(d - {{c}^2} = 0\) will be in the same equivalence class as \((0,0)\). By
algebraic manipulation, we see that \(d = {c}^2\). You may remember from Algebra
I that this is a parabola with its minimum at \((0,0)\), and so the equivalence
class of \((0,0)\) under \(\thicksim\) consists of all points on that particular
parabola. More generaly, if we take any arbitrary point
\((x,y) \in {\reals}^2\), we can find an equivalence class defined by some point
\((0,b) \in {\reals}^2\) by choosing \(b \in \reals\) such that
\(y - {{x}^2} + b = 0\), and this equivalence class will consist of all points
on the parabola defined by \(y = {{x}^2} - b\) (which will have a minimum at
\((0,b)\)).

\end{document}
