\documentclass[main.tex]{subfiles}

\begin{document}

\subproblem{}\label{s2p3b}

Show that \(\ref{s2p2c}\) holds for arbitrary intersections.

\begin{thm}
	The preimage under some function of an arbitrary intersection is equal
	to the arbitrary intersection of preimages under the function.
	Symbolically, that is:
	\[\inverse{f}\left(\Arbitraryintersection{\metaset{B}_i \subset B}{\metaset{B}_i}\right) = \Arbitraryintersection{\metaset{B}_i \subset B}{\inverse{f}\left(\metaset{B}_i\right)}\]
\end{thm}
\begin{proof}
	Let \(A\) and \(B\) be sets, let \(f\) be a mapping from \(A\) to \(B\),
	let \(\metaset{B}\) be a collection of subsets of \(B\), and let
	\(\metaset{B}_i\) be the \(i\)-th member of \(\metaset{B}\).

	First, assume \(x\) is in
	\(\inverse{f}(\arbitraryintersection{\metaset{B}_i \subset B}{\metaset{B}_i})\).
	By the definition of preimage, we know that there exists some \(y\) in
	\(\arbitraryintersection{\metaset{B}_i \subset B}{\metaset{B}_i}\) such
	that \(y = f(x)\). Rephrasing this in terms of \(\metaset{B}\), we can
	equivalently say that \(y\) is in
	\(\arbitraryintersection{\metaset{B}_i \in \metaset{B}}{\metaset{B}_i}\).
	We know that \(y\) is in every set \(\metaset{B}_i\) in \(\metaset{B}\)
	due to the definition of arbitrary intersection. By universal
	instantiation, we'll say that \(y\) is in \(B'\). Recalling that
	\(y = f(x)\), we can also say that \(x\) is in \(\inverse{f}(B')\) due
	to the definition of preimage. By universal generalization, we can say
	that \(x\) is in \(\inverse{f}(\metaset{B}_i)\) for every set
	\(\metaset{B}_i\) in \(\metaset{B}\). By the definition of arbitrary
	intersection, we know that \(x\) is in
	\(\arbitraryintersection{\metaset{B}_i \in \metaset{B}}{\inverse{f}(\metaset{B}_i)}\).
	Rephrasing this in terms of \(B\), we have that \(x\) is in
	\(\arbitraryintersection{\metaset{B}_i \subset B}{\inverse{f}(\metaset{B}_i)}\).
	Thus
	\(\inverse{f}(\arbitraryintersection{\metaset{B}_i \subset B}{\metaset{B}_i})\)
	is a subset of
	\(\arbitraryintersection{\metaset{B}_i \subset B}{\inverse{f}(\metaset{B}_i)}\).

	Next, assume \(x\) is in
	\(\arbitraryintersection{\metaset{B}_i \subset B}{\inverse{f}(\metaset{B}_i)}\).
	Equivalently, we can say that \(x\) is in
	\(\arbitraryintersection{\metaset{B}_i \in \metaset{B}}{\inverse{f}(\metaset{B}_i)}\).
	We know that \(x\) is in the set \(\inverse{f}(\metaset{B}_i)\) for
	every set \(\metaset{B}_i\) in \(\metaset{B}\) due to the definition of
	arbitrary intersection. By universal instantiation, we'll say that \(x\)
	is in \(\inverse{f}(B')\). By the definition of preimage, we know there
	must exist some \(y\) in \(B'\) such that \(y = f(x)\). By universal
	generalization, we can say that \(y\) is in every set \(\metaset{B}_i\)
	in \(\metaset{B}\). We can further say that \(y\) is in
	\(\arbitraryintersection{\metaset{B}_i \in \metaset{B}}{\metaset{B}_i}\)
	by the definition of arbitrary intersection. Rephrasing this in terms of
	\(B\), we can say that \(y\) is in
	\(\arbitraryintersection{\metaset{B}_i \subset B}{\metaset{B}_i}\).
	Recalling that \(y = f(x)\), we can say that \(x\) is in
	\(\inverse{f}(\arbitraryintersection{\metaset{B}_i \subset B}{\metaset{B}_i})\)
	due to the definition of preimage. Thus
	\(\arbitraryintersection{\metaset{B}_i \subset B}{\inverse{f}(\metaset{B}_i)}\)
	is a subset of
	\(\inverse{f}(\arbitraryintersection{\metaset{B}_i \subset B}{\metaset{B}_i})\).

	As we have shown that
	\(\inverse{f}(\arbitraryintersection{\metaset{B}_i \subset B}{\metaset{B}_i})\)
	and
	\(\arbitraryintersection{\metaset{B}_i \subset B}{\inverse{f}(\metaset{B}_i)}\)
	are subsets of each other, we have demonstrated set equality.
\end{proof}

\end{document}
