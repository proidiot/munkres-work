\makeatletter
\def\input@path{{../}}
\makeatother
\documentclass[../main.tex]{subfiles}

\begin{document}

\problem{}
Formulate and prove De Morgan's laws for arbitrary unions and intersections.

\begin{remark}
	All forms of De Morgan's Laws are about the distributivity of a form of
	negation across the equivalent forms of disjunction and conjunction. In
	the first order logical operator sense, that manifests in the
	distribution behavior of \(\neg\) (i.e., logical not) across \(\land\)
	(i.e., logical and) and \(\lor\) (i.e., logical or). As shown earlier in
	this very assignment, the basic set operator sense describes the
	distribution behavior of \(\setminus\) (i.e., basic set difference)
	across \(\cap\) (i.e., basic set intersection) and \(\cup\) (i.e., basic
	set union). When discussing the generalized form of set intersections
	(i.e., \(\bigcap\)) and set unions (i.e., \(\bigcup\)), presumably we
	would be interested in finding the distribution behavior of some
	generalized form of set difference. As we have not discussed such a
	thing to this point in the text or exercises, I will instead proceed
	with the assumption that we want to address the distribution behavior of
	basic set difference across these more generalized operators.
\end{remark}

\subfile{part1/chapter1/section1/problem9/subproblemA.tex}
\subfile{part1/chapter1/section1/problem9/subproblemB.tex}

\end{document}
