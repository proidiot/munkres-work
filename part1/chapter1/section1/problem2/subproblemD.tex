\documentclass[main.tex]{subfiles}

\begin{document}

\subproblem{}\label{2d}
\[A \subset B \lor A \subset C \iff A \subset (B \cap C)\]
\begin{thm}[Double implication fails in~\ref{2d}]
	\(A \subset B \lor A \subset C \centernot\iff A \subset (B \cap C)\)
\end{thm}
\begin{proof}
	For double implication to hold, each predicate must imply the other. We
	will explore whether
	\(A \subset B \lor A \subset C \implies A \subset (B \cap C)\).

	Consider \(A \subset B \lor A \subset C\). Choose
	\(A = \{m | m = 4k, k \in \mathbb{Z}\}\),
	\(B = \{e | e = 2k, k \in \mathbb{Z}\}\),
	and \(C = \{o | o = 2k + 1, k \in \mathbb{Z}\}\) (i.e., \(A\) is the set
	of multiples of	4, \(B\) is the set of all evens, and \(C\) is the set
	of all odds). Since all multiples of 4 are even numbers, it is the case
	that \(A \subset B\). By disjunctive introduction, we can say that
	\(A \subset B \lor A \subset C\).

	Now consider \(B \cap C\). Given our choices for \(B\) and \(C\), this
	would be the intersection of the evens and the odds. Since no even
	number is also an odd number (and vice-versa), we can say that
	\(B \cap C = \emptyset\). Since our choice for \(A\) is the set of
	multiples of 4, we know that \(\exists x \in A\) (for example,
	\(4 = x \in A\)). Further, we can say that
	\(\neg [A \subset (B \cap C)]\) since
	\(\exists x \in A \land x \notin (B \cap C)\).

	Suppose \(A \subset B \lor A \subset C \implies A \subset (B \cap C)\).
	It has already been established that \(A \subset B \lor A \subset C\).
	By modus ponens, \(A \subset (B \cap C)\). However, this contradicts
	\(\neg [A \subset (B \cap C)]\), which has already been established. So
	our supposition is false and
	\(A \subset B \lor A \subset C \centernot\iff A \subset (B \cap C)\).
\end{proof}

\begin{thm}[RHS implies LHS in~\ref{2b}]
	\(A \subset (B \cap C) \implies A \subset B \lor A \subset C\)
\end{thm}
\begin{proof}
	Assume \(A \subset (B \cap C)\). By the definition of the subset
	relation, \(\forall x \in A, x \in (B \cap C)\). By the definition of
	set intersection, this can be extended to
	\(\forall x \in A, x \in B \land x \in C\).  By conjunctive elimination,
	we have that \(\forall x \in A, x \in B\). By the definition of the
	subset relation, \(A \subset B\). By disjunctive introduction,
	\(A \subset B \lor A \subset C\). Therefore
	\(A \subset (B \cap C) \implies A \subset B \lor A \subset C\).
\end{proof}

\end{document}

