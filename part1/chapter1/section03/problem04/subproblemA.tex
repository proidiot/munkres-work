\documentclass[main.tex]{subfiles}

\begin{document}

\subproblem{}\label{s03p04a}

Show that this is an equivalence relation.

\begin{thm}\thlabel{s03p04at1}
	Given \(\therelation{=}\) is an equivalence relation on \(\theset{B}\),
	a surjective function \(\mapping{f}{\theset{A}}{\theset{B}}\), and
	\(\therelation{\sim}\) is a relation on \(\theset{A}\) such that, for
	\(\thepair{a_0}{a_1} \in {\theset{A}^2}\), \({a_0} \sim {a_1}\) if and
	only if \(f(a_0) = f(a_1)\), then \(\therelation{\sim}\) is an
	equivalence relation.
\end{thm}
\begin{proof}
	Let \(\therelation{=}\) be an equivalence relation on \(\theset{B}\).
	Let \(\mapping{f}{\theset{A}}{\theset{B}}\) be a surjective function.
	Let \(\therelation{\sim}\) be a relation on \(\theset{A}\) such that,
	for all \(\thepair{a_0}{a_1} \in {\theset{A}^2}\), \({a_0} \sim {a_1}\)
	iff \(f(a_0) = f(a_1)\). We will show that \(\therelation{\sim}\) is an
	equivalence relation by demonstrating reflexivity, symmetry, and
	transitivity.

	First, let \(x \in \theset{A}\). Note that by the definition of the
	cartesian product, \(\thepair{x}{x} \isin {\theset{A}^2}\). By the rule
	of assignment, there exists some unique \(y \in \theset{B}\) such that
	\(y = f(x)\). By the reflexivity of \(\therelation{=}\) on
	\(\theset{B}\), we have that \(y = y\), and by substitution
	\(f(x) = f(x)\). By the definition of \(\therelation{\sim}\), we have
	that \(x \sim x\). Thus \(\therelation{\sim}\) is reflexive.

	Next, let \(\thepair{x_0}{x_1} \in \therelation{\sim}\). By the
	definition of \(\therelation{\sim}\), we have that \(f(x_0) = f(x_1)\).
	By the rule of assignment, there exists \({y_0} \in \theset{B}\) and
	\({y_1} \in \theset{B}\) such that \({y_0} = f(x_0)\) and
	\({y_1} = f(x_1)\). By substitution, we have that \({y_0} = {y_1}\).
	Since \(\therelation{=}\) is symmetric, we know that \({y_1} = {y_0}\).
	By substitution, \(f(x_1) = f(x_0)\). Again applying the definition of
	\(\therelation{\sim}\), we see that \({x_1} \sim {x_0}\). Thus
	\(\therelation{\sim}\) is symmetric.

	Finally, let \(\thepair{x_0}{x_1} \in \therelation{\sim}\) and
	\(\thepair{x_1}{x_2} \in \therelation{\sim}\). By the definition of
	\(\therelation{\sim}\), we have that \(f(x_0) = f(x_1)\) and
	\(f(x_1) = f(x_2)\). By the rule of assignment, there exists
	\({y_0} \in \theset{B}\), \({y_1} \in \theset{B}\), and
	\({y_2} \in \theset{B}\) such that \({y_0} = f(x_0)\),
	\({y_1} = f(x_1)\), and \({y_2} = f(x_2)\). By substitution, we have
	that \({y_0} = {y_1}\) and \({y_1} = {y_2}\). Since \(\therelation{=}\)
	is transitive, we know that \({y_0} = {y_2}\). By substitution,
	\(f(x_0) = f(x_2)\). Again applying the definition of
	\(\therelation{\sim}\), we see that \({x_0} \sim {x_2}\). Thus
	\(\therelation{\sim}\) is transitive.

	As we have demonstrated the reflexivity, symmetry, and transitivity of
	\(\therelation{\sim}\), we have therefore shown \(\therelation{\sim}\)
	is an equivalence relation.
\end{proof}

\end{document}
