\documentclass[main.tex]{subfiles}

\begin{document}

\subproblem{}\label{s2p4e}

If \(\compose{g}{f}\) is surjective, what can you say about surjectivity of
\(f\) and \(g\)?

\begin{thm}
	Given functions \(\mapping{f}{A}{B}\) and \(\mapping{g}{B}{C}\), if
	\(\compose{g}{f}\) is surjective, then the restriction of \(g\) to the
	image of \(A\) under \(f\) (i.e. \(\restrict{g}{f(A)}\)) must also be
	surjective.
\end{thm}
\begin{proof}
	Let \(A\), \(B\), and \(C\) be sets. Let \(f\) be a mapping from \(A\)
	to \(B\) and let \(g\) be a mapping from \(B\) to \(C\) such that
	\(\compose{g}{f}\) is surjective.

	Assume \(z\) is in \(C\). Since \(\compose{g}{f}\) is surjective, we
	know there exists \(x\) in \(A\) such that
	\(z = \composeapply{g}{f}{x}\). By the definition of function
	composition, we know \(z = g(f(x))\). Since \(f(x)\) is in the image of
	\(A\) under \(f\), we know \(\restrict{g}{f(A)}\) is also surjective.
\end{proof}

\begin{thm}
	Given functions \(\mapping{f}{A}{B}\) and \(\mapping{g}{B}{C}\), if
	\(\compose{g}{f}\) is surjective, then \(f\) is not necessarilly also
	surjective.
\end{thm}
\begin{proof}
	Let \(A\) be \(\{2,3\}\), let \(B\) be \(\{4,5,6\}\), and let \(C\) be
	\(\booleans\). Also, let \(f\) be defined by \(f(x) = 2x\) and let \(g\)
	be defined by \(g(x) = \maybeeqmod{x}{0}{3}\). Note that \(f(2) = 4\) is
	in \(B\) and \(f(3) = 6\) is also in \(B\); since \(2\) and \(3\) are
	the only values in \(A\), we know \(f\) does indeed map from \(A\) to
	\(B\). So we can construct \(\compose{g}{f}\), which is by definition
	\(g(f(x))\) for \(x\) in \(A\). Now let \(z\) be in \(C\). So either
	\(z = \true\) or \(z = \false\), and we will explore these cases
	separately.

	First, assume \(z = \true\). Choose \(x = 3\), which is in \(A\). Now
	\(f(x) = 2 \times 3 = 6\). By substitution, we see that
	\(\composeapply{g}{f}{x} = g(f(x)) = g(6) = \maybeeqmod{6}{0}{3} = \true\).
	So there exists \(x\) in \(A\) such that \(z = \composeapply{g}{f}{x}\).

	Next, assume \(z = \false\). Choose \(x = 2\), which is in \(A\). Now
	\(f(x) = 2 \times 2 = 4\). By substitution, we see that
	\(\composeapply{g}{f}{x} = g(f(x)) = g(4) = \maybeeqmod{4}{0}{3} = \false\).
	So there exists \(x\) in \(A\) such that \(z = \composeapply{g}{f}{x}\).

	Since for every \(z\) in \(C\) there exists \(x\) in \(A\) such that
	\(z = \composeapply{g}{f}{x}\), we have demonstrated that
	\(\compose{g}{f}\) is surjective.

	Now assume \(y = 5\), which is in \(B\). Suppose there was an \(x\) in
	\(A\) such that \(y = f(x)\). By substitution, we see that
	\(5 = f(x) = 2x\). Solving for \(x\), we get \(x = \frac{5}{2}\).
	However, observe that \(\frac{5}{2}\) is not in \(A\), which contradicts
	our assumption that the domain of \(f\) is \(A\). So our supposition is
	false, and there is no \(x\) in \(A\) such that \(y = 5 = f(x)\). Thus
	\(f\) is not surjective.

	Therefore it is possible for \(\compose{g}{f}\) to be surjective even
	though \(f\) is not surjective.
\end{proof}

\end{document}
