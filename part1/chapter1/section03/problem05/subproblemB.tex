\documentclass[main.tex]{subfiles}

\begin{document}

\subproblem{}\label{s03p05b}

Show that given any collection of equivalence relations on a set \(\theset{A}\),
their intersection is an equivalence relation on \(\theset{A}\).

\begin{lemma}\thlabel{emptyIntersectNotEquivRel}
	Given an empty collection of equivalence relations on a non-empty set
	\(\theset{A}\), the intersection of the empty collection is not an
	equivalence relation on \(\theset{A}\).
\end{lemma}
\begin{proof}
	Assume there is a non-empty set \(\theset{A}\) and an empty set
	\(\theset{E}\) of equivalence relations on \(\theset{A}\). Since
	\(\theset{E}\) has no elements, then \(\intersectionofset{E}\) is the
	empty set. Since \(\theset{A}\) is non-empty, let \(a\) be some element
	of \(\theset{A}\). Suppose \(\intersectionofset{E}\) is an
	equivalence relation on \(\theset{A}\). By the definition of an
	equivalence relation on \(\theset{A}\), \(\intersectionofset{E}\) is a
	reflexive relation on \(\theset{A}\). By the definition of a reflexive
	relation on \(\theset{A}\), we know that \(\thepair{a}{a}\) must be in
	\(\intersectionofset{E}\). However, this contradicts
	\(\intersectionofset{E}\) being the empty set. So the supposition is
	false, and therefore \(\intersectionofset{E}\) is not an equivalence
	relation on \(\theset{A}\).
\end{proof}

\begin{lemma}\thlabel{relIntersectIsRel}
	Given a collection consisting of relations on a set \(\theset{A}\),
	their intersection is a relation on \(\theset{A}\).
\end{lemma}
\begin{proof}
	Assume there is a set \(\theset{A}\) and a non-empty set \(\theset{E}\)
	consisting of relations on \(\theset{A}\). Let \(\theset{E_0}\) be a
	member of \(\theset{E}\). By the definition of \(\theset{E}\), we know
	that \(\theset{E_0}\) is a relation on \(\theset{A}\). By the definition
	of a relation on \(\theset{A}\), we know that \(\theset{E_0}\) is a
	subset of \(\theset{A}^2\). By definition, we know that
	\(\intersectionofset{E}\) is a subset of \(\theset{E_0}\). By the
	transitivity of the subset relation, we know that
	\(\intersectionofset{E}\) is a subset of \(\theset{A}^2\). Therefore
	\(\intersectionofset{E}\) is a relation on \(\theset{A}\) by definition.
\end{proof}

\begin{lemma}\thlabel{reflIntersectIsRefl}
	Given a non-empty collection consisting of reflexive relations on a set
	\(\theset{A}\), their intersection is a reflexive relation on
	\(\theset{A}\).
\end{lemma}
\begin{proof}
	Assume there is a set \(\theset{A}\) and a non-empty set \(\theset{E}\)
	consisting of reflexive relations on \(\theset{A}\). Suppose
	\(\intersectionofset{E}\) is not a reflexive relation on \(\theset{A}\).
	Then by the definition of a reflexive relation, there exists some
	element \(a\) in \(\theset{A}\) such that \(\thepair{a}{a}\) is not in
	\(\intersectionofset{E}\). By definition, there is at least one member
	\(\theset{E_0}\) in \(\theset{E}\) such that \(\thepair{a}{a}\) is not
	in \(\theset{E_0}\). Note that by the definition of \(\theset{E}\), we
	know that \(\theset{E_0}\) is a reflexive relation on \(\theset{A}\). So
	by the definition of a reflexive relation on \(\theset{A}\), we can say
	that \(\thepair{a}{a}\) is in \(\theset{E_0}\). However, this
	contradicts the statement that \(\thepair{a}{a}\) is not in
	\(\theset{E_0}\). So the supposition was false, and
	\(\intersectionofset{E}\) is a reflexive relation on \(\theset{A}\).
\end{proof}

\begin{lemma}\thlabel{symmIntersectIsSymm}
	Given a collection consisting of symmetric relations on a set
	\(\theset{A}\), their intersection is a symmetric relation on
	\(\theset{A}\).
\end{lemma}
\begin{proof}
	Assume there is a set \(\theset{A}\) and a set \(\theset{E}\) consisting
	of symmetric relations on \(\theset{A}\). Suppose
	\(\intersectionofset{E}\) is not a symmetric relation on \(\theset{A}\).
	Then by definition, there exists \(\somepair{a}{b}\) in
	\(\intersectionofset{E}\) such that \(a \isin \theset{A}\) and
	\(b \isin \theset{A}\) but \(\thepair{b}{a}\) is not in
	\(\intersectionofset{E}\). So \(\thepair{a}{b}\) is in every member of
	\(\theset{E}\), but there exists \(\someset{E_0}\) in \(\theset{E}\)
	such that \(\thepair{b}{a}\) is not in \(\theset{E_0}\). As we have
	established that \(\thepair{a}{b}\) is in every member of
	\(\theset{E}\), we know that \(\thepair{a}{b}\) is in \(\theset{E_0}\).
	By the definition of \(\theset{E}\), we can say that \(\theset{E_0}\) is
	a symmetric relation on \(\theset{A}\). Since \(\thepair{a}{b}\) is in
	\(\theset{E_0}\), we know that \(\thepair{b}{a}\) is also in
	\(\theset{E_0}\) by the definition of a symmetric relation. However,
	this contradicts the statement that \(\thepair{b}{a}\) is not in
	\(\theset{E_0}\). So the supposition was false, and
	\(\intersectionofset{E}\) is a symmetric relation on \(\theset{A}\).
\end{proof}

\begin{lemma}\thlabel{transIntersectIsTrans}
	Given a collection consisting of transitive relations on a set
	\(\theset{A}\), their intersection is a transitive relation on
	\(\theset{A}\).
\end{lemma}
\begin{proof}
	Assume there is a set \(\theset{A}\) and a non-empty set \(\theset{E}\)
	consisting of transitive relations on \(\theset{A}\). Suppose
	\(\intersectionofset{E}\) is not a transitive relation on
	\(\theset{A}\). Then by definition, there exists
	\(\somepair{a}{b} \in \intersectionofset{E}\) and
	\(\somepair{b}{c} \in \intersectionofset{E}\) such that
	\(a \isin \theset{A}\), that \(b \isin \theset{A}\), and that
	\(c \isin \theset{A}\), but \(\thepair{a}{c}\) is not in
	\(\intersectionofset{E}\). So \(\thepair{a}{b}\) and \(\thepair{b}{c}\)
	are in every member of \(\theset{E}\), but there exists
	\(\someset{E_0}\) in \(\theset{E}\) such that \(\thepair{a}{c}\) is not
	in \(\theset{E_0}\). Now by the definition of \(\theset{E}\), we know
	that \(\theset{E_0}\) is a transitive relation on \(\theset{A}\). As we
	have already established that both
	\(\thepair{a}{b} \isin \theset{E_0}\) and
	\(\thepair{b}{c} \isin \theset{E_0}\), we can say that
	\(\thepair{a}{c} \isin \theset{E_0}\) by the definition of a transitive
	relation on \(\theset{A}\). However, this contradicts the statement that
	\(\thepair{a}{c}\) is not in \(\theset{E_0}\). So the supposition was
	false, and \(\intersectionofset{E}\) is a transitive relation on
	\(\theset{A}\).
\end{proof}

\begin{thm}\thlabel{equivIntersectIsEquiv}
	Given a non-empty collection of equivalence relations on a set
	\(\theset{A}\), their intersection is an equivalence relation on
	\(\theset{A}\).
\end{thm}
\begin{proof}
	Assume there is a set \(\theset{A}\) and a non-empty set \(\theset{E}\)
	consisting of equivalence relations on \(\theset{A}\). As
	\(\intersectionofset{E}\) is a relation on \(\theset{A}\) (by
	\thref{relIntersectIsRel}) that is reflexive (by
	\thref{reflIntersectIsRefl}), symmetric (by
	\thref{symmIntersectIsSymm}), and transitive (by
	\thref{transIntersectIsTrans}), we can say that
	\(\intersectionofset{E}\) is an equivalence relation on \(\theset{A}\)
	by definition.
\end{proof}

\end{document}
