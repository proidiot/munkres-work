\documentclass[main.tex]{subfiles}

\begin{document}

\subproblem{}\label{5a}

\(x \isin \arbitraryunion{A \in \metaset{A}}{A} \implies x \isin A\) for at
least one \(A \in \metaset{A}\).

\begin{thm}[Implication holds in~\ref{5a}]\thlabel{5at}
	Given \(\metaset{A}\) is a nonempty collection of sets, for every \(x\)
	in the generalized union over all sets in \(\metaset{A}\), there exists
	some set \(A\) in \(\metaset{A}\) such that \(x\) is in \(A\).
	Symbolically, that is
	\[x \isin \Arbitraryunion{A \in \metaset{A}}{A} \implies x \isin \setbuild{x'}{\thereexists A \in \metaset{A} \suchthat x' \isin A}\]
\end{thm}
\begin{proof}
	By the definition of the generalized union, every member of a
	generalized union over a collection of sets is a member of at least one
	of the constituent sets of the generalized union.
\end{proof}

\begin{thm}[Implication holds for the converse of~\ref{5a}]\thlabel{5atc}
	Given \(\metaset{A}\) is a nonempty collection of sets, for every \(x\)
	such that there exists some set \(A\) in \(\metaset{A}\) where \(x\) is
	in \(A\), it is the case that \(x\) is also in the generalized union
	over all sets in \(\metaset{A}\). Symbolically, that is
	\[x \isin \setbuild{x'}{\thereexists A \in \metaset{A} \suchthat x' \isin A} \implies x \isin \Arbitraryunion{A \in \metaset{A}}{A}\]
\end{thm}
\begin{proof}
	By the definition of the generalized union, any item that is a member of
	at least one constituent set of a generalized union is also a member of
	the generalized union.
\end{proof}


\end{document}
