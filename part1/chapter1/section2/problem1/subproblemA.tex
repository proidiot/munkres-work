\documentclass[main.tex]{subfiles}

\begin{document}

\subproblem{}\label{s2p1a}

Show that \(A_0 \subset \inverse{f}(f(A_0))\) and that equality holds if \(f\)
is injective.

\begin{thm}\thlabel{s2p1at1}
	Given a function \(f\) which maps \(A\) to \(B\), and given a subset
	\(A_0\) of \(A\), then \(A_0\) is a subset of the preimage of the image
	of \(A_0\) under \(f\). Symbolically, that is:
	\[A_0 \subset \inverse{f}(f(A_0))\]
\end{thm}
\begin{proof}
	Let \(A\) and \(B\) be sets, let \(f\) be a function mapping \(A\) to
	\(B\), and let \(A_0\) be a subset of \(A\). Assume \(x\) is some
	arbitrary element of \(A_0\). Consider the image of \(A_0\) under \(f\);
	we'll call it \(B'\). By definition, we know that \(B'\) is the set of
	all values of \(b = f(a)\) for at least one \(a\) in \(A_0\). Since
	\(x\) is in \(A_0\), we know that the value of \(f(x)\) is in \(B'\);
	we'll call that value \(y\) (i.e., \(y = f(x) \in B'\)). Next, consider
	the preimage of \(B'\) under \(f\); we'll call it \(A'\). By definition,
	we know that \(A'\) is the set of all values \(a\) such that \(f(a)\) is
	in \(B'\). Recall that \(y\) is in \(B'\). Since \(x\) is a value such
	that \(y = f(x)\) is in \(B'\), we can see that \(x\) is in \(A'\). By
	the definition of image and preimage, we have shown that \(x\) is in
	\(\inverse{f}(f(A_0))\), and so \(A_0\) is a subset of
	\(\inverse{f}(f(A_0))\).
\end{proof}

\begin{thm}\thlabel{s2p1at2}
	Given a function \(f\) which maps \(A\) to \(B\), and given a subset
	\(A_0\) of \(A\), and given injectivity of \(f\), then \(A_0\) is equal
	to the preimage of the image of \(A_0\) under \(f\). Symbolically, that
	is:
	\[A_0 = \inverse{f}(f(A_0))\]
\end{thm}
\begin{proof}
	Let \(A\) and \(B\) be sets, let \(f\) be a function mapping \(A\) to
	\(B\), let \(A_0\) be a subset of \(A\), and let \(f\) be injective.
	Note that \(A_0\) is a subset of \(\inverse{f}(f(A_0))\) since the
	assumptions of~\thref{s2p1at1} are a weaker subset of our current
	assumptions.

	For brevity, we'll refer to \(\inverse{f}(f(A_0))\) as \(A'\) and
	\(f(A_0)\) as \(B'\) (so \(B'\) is the image of \(A_0\) under \(f\) and
	\(A'\) is the preimage of \(B'\) under \(f\)). Assume \(x\) is an
	arbitrary member of \(A'\). By the definition of preimage, we know that
	\(f(x)\) is in \(B'\); we'll refer to the particular value of \(f(x)\)
	as \(y\) (so \(y\) is in \(B'\)). Now the definition of the image of
	\(A_0\) under \(f\) gives us that \(B'\) is the set of all values \(b\)
	such that \(b = f(a)\) for at least one value of \(a\) in \(A_0\). Since
	\(y\) is in \(B'\), we know there is at least one value \(x'\) in
	\(A_0\) such that \(y = f(x')\), but considering only the definition of
	image, we would not know if this particular value \(x'\) is unique.
	However, since we have indeed established that \(f\) is injective, by
	definition we know that \(f(x') = y = f(x)\) implies that \(x' = x\). So
	we know that \(x = x'\) is in \(A_0\). Thus we see that
	\(\inverse{f}(f(A_0))\) is a subset of \(A_0\).

	Since we have established both that \(A_0\) is a subset of
	\(\inverse{f}(f(A_0))\) and that \(\inverse{f}(f(A_0))\) is a subset of
	\(A_0\), we have established equality.
\end{proof}

\end{document}
