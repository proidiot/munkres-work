\makeatletter
\def\input@path{{../}}
\makeatother
\documentclass[../main.tex]{subfiles}

\begin{document}

\problem{}\label{s2p6}

Let \(\mapping{f}{\reals}{\reals}\) be the function \(f(x) = x^3 - x\). By
restricting the domain and range of \(f\) appropriately, obtain from \(f\) a
bijective function \(g\). Draw the graphs of \(g\) and \(\inverse{g}\). (There
are several possible choices for \(g\).)

\begin{thm}
	Let \(\mapping{f}{\reals}{\reals}\) be defined by \(f(x) = x^3 - x\).
	Given \(A\) and \(B\) are closed intervals on \(\reals\) such that
	\(A = [-1/\sqrt{3},1/\sqrt{3}]\) and \(B = [-2/3\sqrt{3},2/3\sqrt{3}]\),
	it is the case that \(\mapping{g}{A}{B}\) defined by the restriction of
	\(f\) to \(A\) is bijective.
\end{thm}
\todo{}
%\begin{proof}
%	Let \(f\) be a mapping from \(\reals\) to \(\reals\) defined by
%	\(f(x) = x^3 - x\), let \(A\) be the set consisting of the closed
%	interval of \(\reals\) between \(-1/\sqrt{3}\) and \(1/\sqrt{3}\),
%	let \(B\) be the set consisting of the closed interval of \(\reals\)
%	between \(-2/(3\sqrt{3})\) and \(2/(3\sqrt{3})\), and let \(g\) be a
%	mapping from \(A\) to \(B\) defined by the restriction of \(f\) to
%	\(A\).
%
%	First, we will determine if \(g\) is injective. Assume \(x_0\) and
%	\(x_1\) are in \(A\) such that \(f(x_0)\) and \(f(x_1)\) are equal. Note
%	that we immediately have \(f(x_1) - f(x_0) = 0\) by algebraic
%	manipulation. To demonstrate the injectivity of \(g\), we simply need to
%	show that the equality of \(x_0\) and \(x_1\) follows. Suppose \(x_0\)
%	and \(x_1\) are not equal. By the closure of \(\reals\) under
%	subtraction, there exists \(d\) in \(\reals\) such that
%	\(d = x_1 - x_0\). Since we have supposed that \(x_0 \neq x_1\), it
%	follows that \(d\) is non-zero. By algebraic manipulation, we see that
%	\(x_0 + d = x_1\). Recall that \(f(x_1) = {x_1}^3 - {x_1}\) by the
%	definition of \(f\). By substitution, we have that
%	\begin{align*}
%		f(x_1) &= {x_1}^3 - {x_1} \\
%		       &= (x_0 + d) ({(x_0 + d)}^2 - 1) \\
%		       &= {(x_0 + d)}^3 - (x_0 + d) \\
%		       &= ({x_0}^3 + 3 {x_0}^2 d + 3 {x_0} d^2 + d^3) - (x_0 + d) \\
%		       &= ({x_0}^3 - {x_0}) + (3 {x_0}^2 d + 3 {x_0} d^2 + d^3 - d) \\
%		       &= f(x_0) + (d^3 + 3 {x_0} d^2 + 3 {x_0}^2 d - d) \\
%		       &= f(x_0) + d (d^2 + 3 {x_0} d + 3 {x_0}^2 - 1)
%	\end{align*}. Note that \begin{align*}
%		f(x_1) - f(x_0) &= f(x_0) + d (d^2 + 3 {x_0} d + 3 {x_0}^2 - 1) - f(x_0) \\
%		                &= d (d^2 + 3 {x_0} d + 3 {x_0}^2 - 1)
%	\end{align*}. Since we already know that \(f(x_1) - f(x_0) = 0\), it
%	must be that \(d (d^2 + 3 {x_0} d + 3 {x_0}^2 - 1) = 0\). Since a zero
%	mulitplicative product requires at least one multiplicand to be zero,
%	and since we have already established that \(d\) is non-zero, it must be
%	that \(d^2 + 3 {x_0} d + 3 {x_0}^2 - 1 = 0\). By algebraic manipulation,
%	we immediately see \(d^2 + 3 {x_0} d + 3 {x_0}^2 = 1\).
%	
%	In order to
%	determine whether \(x_0\) and \(x_1\) are equal, we can consider the
%	difference \(d = x_0 - x_1\) (which would be in \(\reals\) by closure
%	under subtraction), noting that \(d = 0\) would imply equality. Also
%	note that \(x_0 = x_1 + d\) by algebraic manipulation. Now applying
%	\(f\), we have that \(f(x_0) = {x_0}^3 - x_0\) and
%	\(f(x_1) = {x_1}^3 - x_1\).  and so \({x_0}^3 - x_0 = {x_1}^3 - x_1\) by
%
%
%
%
%
%
%
%	\(f(x_1) = {x_1}^3 - x_1\), and so \({x_0}^3 - x_0 = {x_1}^3 - x_1\) by
%	substitution. Further, we see that
%	\({x_0}^3 - x_0 = (x_1 + d)^3 - (x_1 + d) = {x_1}^3 + {x_1}^2 d + {x_1} d^2 + d^3 - x_1 - d\).
%	Since \({x_0}^3 - x_0 = {x_1}^3 - x_1\), substitution gives us
%	\({x_1}^3 + {x_1}^2 d + {x_1} d^2 + d^3 - x_1 - d = {x_1}^3 - x_1\).
%	Subtracting \({x_1}^3 - {x_1}\) from both sides of this equation, we see
%	that \({x_1}^2 d + {x_1} d^2 + d^3 - d = 0\). Suppose \(d \neq 0\).
%	Factoring out \(d\), we have that \(d (d^2 + {x_1} d + {x_1}^2 - 1) = 0\).
%	As a product with zero value implies at least one multiplicand is zero,
%	and since we have stated that \(d \neq 0\), it must be the case that
%	\(d^2 + {x_1} d + {x_1}^2 - 1 = 0\). Using the
%	quadratic formula, we have that
%	\(d = (-{x_1} \plusorminus \sqrt{{x_1}^2 - 4(1)({x_1}^2 - 1)})/(2(1))\).
%	By simplification, we have that \(d = (-{x_1} \plusorminus \sqrt{-3{x_1}^2 + 4})/2 = (-{x_1} \plusorminus \sqrt{-1({x_1}+2)({x_1}-2)})/2\).\todo{}
%
%	Next, we will determine if \(g\) is surjective. Assume \(y\) is in
%	\(B\). \todo{}
%
%	As we have demonstrated that \(g\) is both injective and surjective, we
%	have therefore shown that \(g\) is bijective by definition.
%\end{proof}

\begin{figure}[H]
\centering{}
\resizebox{\textwidth}{!}{%
\begin{tikzpicture}[baseline]
	\begin{axis}[
		xlabel={\(x\)},
		ylabel={\(y\)},
		axis lines=middle,
		xmin=-0.66, xmax=0.66,
		ymin=-0.66, ymax=0.66,
		minor y tick num=0.2,
	]
		\addplot
		[black, domain=-1/sqrt(3):1/sqrt(3), samples=200]
			{x^3 - x}
		node[below right, pos=0]
			{\(g(x)\)};
	\end{axis}
\end{tikzpicture} %
\hskip 10 pt %
\begin{tikzpicture}[baseline]
	\begin{axis}[
		xlabel={\(x\)},
		ylabel={\(y\)},
		axis lines=middle,
		xmin=-0.66, xmax=0.66,
		ymin=-0.66, ymax=0.66,
		minor y tick num=0.2,
	]
		\addplot
		[black, domain=-1/sqrt(3):1/sqrt(3), samples=200]
			({x^3 - x},{x})
		node[below left, pos=1]
			{\(\inverse{g}(x)\)};
	\end{axis}
\end{tikzpicture}%
}
\caption{The bijective function \(g\) defined by the restriction of \(f\) to the closed interval bounded by \(\pm 1/\sqrt{3}\), and the inverse of \(g\).}
\end{figure}

\end{document}
