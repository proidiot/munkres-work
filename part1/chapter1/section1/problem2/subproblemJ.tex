\documentclass[main.tex]{subfiles}

\begin{document}

\subproblem{}\label{2j}
\[A \subset C \land B \subset D \implies (A \times B) \subset (C \times D)\]
\begin{thm}[Implication holds in~\ref{2j}]
	Given sets \(A\), \(B\), \(C\), and \(D\), if \(A\) is a subset of \(C\)
	and \(B\) is a subset of \(D\), then the cartesian product of \(A\) and
	\(B\) is a subset of the cartesian product of \(C\) and \(D\).
	Symbolically, that is
	\[A \subset C \land B \subset D \implies (A \times B) \subset (C \times D)\]
\end{thm}
\begin{proof}
	Let \(A\), \(B\), \(C\), and \(D\) be sets such that \(A\) is a subset
	of \(C\) and \(B\) is a subset of \(D\). Consider the nature of
	\(A \times B\). If the set is empty, then it is of course a subset of
	\(C \times D\) since the empty set is a subset of all sets. If instead
	\(A \times B\) is not empty, we can let \((x, y)\) be some arbitrary
	member of \(A \times B\) where \(x\) is in \(A\) and \(y\) is in \(B\).
	Since we know that \(A\) is a subset of \(C\) and \(B\) is a subset of
	\(D\), we can say that \(x\) is in \(C\) and that \(y\) is in \(D\).
	Given the definition of cartesian product, we can say that \((x, y)\) is
	also a member of \(C \times D\). Therefore \(A \times B\) must be a
	subset of \(C \times D\).
\end{proof}

\end{document}

