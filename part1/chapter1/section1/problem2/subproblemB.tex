\documentclass[main.tex]{subfiles}

\begin{document}

\subproblem{}\label{2b}
\[A \subset B \lor A \subset C \iff A \subset (B \cup C)\]
\begin{thm}[Double implication fails in~\ref{2b}]
	\(A \subset B \lor A \subset C \centernot\iff A \subset (B \cup C)\)
\end{thm}
\begin{proof}
	For double implication to hold, each predicate must imply the other. We
	will explore whether
	\(A \subset (B \cup C) \implies A \subset B \lor A \subset C\).

	Consider \(A \subset (B \cup C)\). Choose
	\(A = \{m | m = 3k, k \in \mathbb{Z}\}\),
	\(B = \{e | e = 2k, k \in \mathbb{Z}\}\),
	and \(C = \{o | o = 2k + 1, k \in \mathbb{Z}\}\) (i.e., \(A\) is the set
	of multiples of	3, \(B\) is the set of all evens, and \(C\) is the set
	of all odds). Since all integers are even or odd, we can say that it is
	indeed the case that \(A \subset (B \cup C)\) holds for our choices of
	\(A\), \(B\), and \(C\). Suppose
	\(A \subset (B \cup C) \implies A \subset B \lor A \subset C\). Since
	we have already shown that the antecedent holds for our choices of
	\(A\), \(B\), and \(C\), this would require the consequent to be true as
	well. So either \(A \subset B\) or \(A \subset C\) or both.

	First consider the possibility that \(A \subset B\). This would mean
	that the set of multiples of 3 is a subset of the evens. This would
	imply that \(\forall x \in A, x \in B\). Now note that \(3 = x \in A\)
	because 3 is a multiple of 3. However, \(3 = x \notin B\) as 3 is not
	even. This contradicts our requirement that for any \(x \in A\), it is
	also the case that \(x \in B\). So our supposition would be false in the
	case that \(A \subset B\), and the implication would not hold.

	Next consider the possibility that \(A \subset C\). This would mean
	that the set of multiples of 3 is a subset of the odds. This would imply
	that \(\forall x \in A, x \in C\). Now note that \(6 = x \in A\) because
	6 is a multiple of 3. However, \(6 = x \notin C\) as 6 is not odd. This
	contradicts our requirement that for any \(x \in A\), it is also the
	case that \(x \in C\). So our supposition would be false in the case
	that \(A \subset C\), and the implication would not hold.

	Since both cases derived from our supposition reach contradictions, our
	supposition was false and the implication therefore does not hold. Since
	we have demonstrated that one of the predicates does not imply the other
	predicate, we have therefore established that the double implication
	fails as well.
\end{proof}

\begin{thm}[LHS implies RHS in~\ref{2b}]
	\(A \subset B \lor A \subset C \implies A \subset (B \cup C)\)
\end{thm}
\begin{proof}
	It is given that either \(A \subset B\) or \(A \subset C\) or both.

	Consider the possibility that \(A \subset B\). It is either the case
	that \(A = \emptyset\) or \(\exists x \in A \implies x \in B\). In the
	trivial case that \(A = \emptyset\), \(A \subset (B \cup C)\) since the
	empty set is a subset of all sets. In the case that \(A\) is not empty,
	then we can say that \(\exists x \in A \implies (x \in B \lor x \in C)\)
	by disjunctive inclusion. By the definition of set union, we have that
	\(x \in (B \cup C)\). So
	\(A \neq \emptyset \implies A \subset (B \cup C)\) in this case. A
	similar construction can be made given \(A \subset C\) to show that
	\(A \subset (B \cup C)\) regardless of whether \(A = \emptyset\) as
	well. Since \(A \subset (B \cup C)\) in all cases derived from our
	assumptions, we have therefore demonstrated that
	\(A \subset B \lor A \subset C \implies A \subset (B \cup C)\).
\end{proof}


\end{document}

