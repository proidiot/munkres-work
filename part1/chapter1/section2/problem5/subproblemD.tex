\documentclass[main.tex]{subfiles}

\begin{document}

\subproblem{}\label{s2p5d}

Can a function have more than one left inverse? More than one right inverse?

\begin{thm}
	A function can have more than one left inverse.
\end{thm}
\begin{proof}
	Let \(A\) be the set \(\{0,1\}\), let \(B\) be the set \(\{0,1,3\}\),
	let \(\mapping{f}{A}{B}\) be defined by \(f(x) = x\), let
	\(\mapping{g}{B}{A}\) be defined by \(g(y) = y \mod 2\), and let
	\(\mapping{h}{B}{A}\) be defined by \(h(y) = y \mod 3\). Note that for
	their given domains, each of these functions produce only outputs in
	their given codomains.

	First, let's determine if \(g\) is a left inverse of \(f\). Assume \(x\)
	is in \(A\). We know \(\composeapply{g}{f}{x} = g(f(x))\) by definition.
	By substitution, we have that
	\(\composeapply{g}{f}{x} = g(f(x)) = (f(x)) \mod 2 = x \mod 2\).
	Noting that \(x < 2\) for all \(x\) in \(A\), we can simplify this
	expression to \(\composeapply{g}{f}{x} = x\). So we know
	\(\compose{g}{f} = i_A\) by the definition of the identity function, and
	thus \(g\) is a left inverse of \(f\).

	Next, let's determine if \(h\) is a left inverse of \(f\). Assume \(x\)
	is in \(A\). We know \(\composeapply{h}{f}{x} = h(f(x))\) by definition.
	By substitution, we have that
	\(\composeapply{h}{f}{x} = h(f(x)) = (f(x)) \mod 3 = x \mod 3\).
	Noting that \(x < 3\) for all \(x\) in \(A\), we can simplify this
	expression to \(\composeapply{h}{f}{x} = x\). So we know
	\(\compose{h}{f} = i_A\) by the definition of the identity function, and
	thus \(h\) is a left inverse of \(f\).

	Finally, we must determine if \(g\) and \(h\) are distinct. We have
	prevously shown that \(g(y)\) and \(h(y)\) have the same behavior for
	\(y = 0\) and \(y = 1\), but \(y = 3\) is also in \(B\), so choose
	\(y' = 3\). Now \(g(y') = g(3) = 3 \mod 2 = 1\) and
	\(h(y') = h(3) = 3 \mod 3 = 0\). Since \(g(y') = 1 \neq 0 = h(y')\) for
	a valid choice of \(y'\) in \(B\), we see that \(g\) and \(h\) are
	distinct. As we have already demonstrated that each of \(g\) and \(h\)
	are left inverses of \(f\) despite \(g\) and \(h\) being distinct, it is
	therefore the case that a function can have more than one left inverse.
\end{proof}

\begin{thm}\thlabel{s2p5dt2}
	A surjective function cannot have more than one left inverse.
\end{thm}
\begin{proof}
	Let \(A\) and \(B\) be sets, and let \(\mapping{f}{A}{B}\) be a
	surjective function with a left inverse \(g\).

	Suppose there is another left inverse \(h\) of \(f\) such that
	\(g \neq h\). By the definition of left inverse, we know
	\(\composeapply{h}{f}{x} = x\) for all \(x\) in \(A\). By the definition
	of function composition, we have that \(h(f(x)) = x\) for all \(x\) in
	\(A\). Recalling that \(g\) is also a left inverse of \(f\), we see that
	\(h(f(x)) = x = g(f(x))\) for all \(x\) in \(A\). So we see that
	\(h(f(A)) = g(f(A))\) by the definition of image under a function. Since
	\(g \neq h\), there must be some \(y\) in \(B\) such that
	\(g(y) \neq h(y)\), and as we have established that
	\(g(f(A)) = h(f(A))\), it must be that \(y\) is in \(B \without f(A)\).
	However, we have assumed that \(f\) is surjective, we know there is no
	\(y\) in \(B \without f(A)\). Therefore our supposition is false, and a
	surjective function \(f\) with a left inverse \(g\) has no additional
	left inverse \(h\) which is distinct from \(g\).
\end{proof}

\begin{thm}
	A function can have more than one right inverse.
\end{thm}
\begin{proof}
	Let \(A\) be the set \(\{0,1,2,3\}\), let \(B\) be the set \(\{0,1\}\),
	let \(\mapping{f}{A}{B}\) be defined by \(f(x) = x \mod 2\), let
	\(\mapping{g}{B}{A}\) be defined by \(g(y) = y\), and let
	\(\mapping{h}{B}{A}\) be defined by \(h(y) = y + 2\).  Note that for
	their given domains, each of these functions produce only outputs in
	their given codomains.

	First, let's determine if \(g\) is a right inverse of \(f\). Assume
	\(y\) is in \(B\). We know \(\composeapply{f}{g}{y} = f(g(y))\) by
	definition. By substitution we have that
	\(\composeapply{f}{g}{y} = f(g(y)) = (g(y)) \mod 2 = y \mod 2\). Noting
	that \(y < 2\) for all \(y\) in \(B\), we can simplify this expression
	to \(\composeapply{f}{g}{y} = y\). So we know \(\compose{f}{g} = i_B\)
	by the definition of the identity function, and thus \(g\) is a right
	inverse of \(f\).

	Next, let's determine if \(h\) is a right inverse of \(f\). Assume \(y\)
	is in \(B\). We know \(\composeapply{f}{h}{y} = f(h(y))\) by definition.
	By substitution we have that
	\(\composeapply{f}{h}{y} = f(h(y)) = (h(y)) \mod 2 = (y + 2) \mod 2\).
	By the definition of the modulus, we know \(\eqmod{y + 2}{y}{2}\), and
	so we can simplify the previous equation to
	\(\composeapply{f}{h}{y} = y \mod 2\). Again, since \(y < 2\) for all
	\(y\) in \(B\), we can further simplify to
	\(\composeapply{f}{h}{y} = y\). So we know \(\compose{f}{h} = i_B\) by
	the definition of the identity function, and thus \(h\) is a right
	inverse of \(f\).

	Finally, we must determine if \(g\) and \(h\) are distinct. Choose
	\(y = 0\), which is in \(B\). We see that \(g(y) = g(0) = 0\) and
	\(h(y) = h(0) = 0 + 2 = 2\). Since \(g(y) = 0 \neq 2 = h(y)\) for some
	valid choice of \(y\) in \(B\), we have that \(g\) and \(h\) are
	distinct. Therefore it is possible for a function to have more than one
	right inverse.
\end{proof}

\begin{thm}\thlabel{s2p5dt4}
	An injective function cannot have more than one right inverse.
\end{thm}
\begin{proof}
	Let \(A\) and \(B\) be sets, and let \(\mapping{f}{A}{B}\) be an
	injective function with a right inverse \(g\).

	Suppose there is another right inverse \(h\) of \(f\) such that
	\(g \neq h\). So there exists some \(y\) in \(B\) such that
	\(g(y) \neq h(y)\). By the definition of right inverse, we know
	\(\composeapply{f}{h}{y} = y\). By the definition of function
	composition, we have that \(f(h(y)) = y\). Recalling that \(g\) is also
	a right inverse of \(f\), we see that \(f(h(y)) = y = f(g(y))\).
	However, since \(f\) is injective, we see that \(f(h(y)) = f(g(y))\)
	implies \(h(y) = g(y)\), contradicting \(g(y) \neq h(y)\) which was
	shown from our supposition. Therefore our supposition is false, and
	given an injective function \(f\) with a right inverse \(g\), there does
	not exist a right inverse \(h\) of \(f\) such that \(h\) is distinct
	from \(g\).
\end{proof}

\end{document}
