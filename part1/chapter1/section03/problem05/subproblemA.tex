\documentclass[main.tex]{subfiles}

\begin{document}

\subproblem{}\label{s03p05a}

Show that \(\theset{S'}\) is an equivalence relation on the real line and
\(\theset{S'} \isasupersetof \theset{S}\). Describe the equivalence classes of
\(\theset{S'}\).

\begin{thm}\thlabel{s03p05at1}
	\(\theset{S'}\) is reflexive.
\end{thm}
\begin{proof}
	Assume \(\theset{S'}\) is the set of all \(\thepair{x}{y}\) in
	\(\reals ^ 2\) such that \(y - x\) is an integer. Let \(r\) be some
	value in \(\reals\). Note that \(\thepair{r}{r}\) is in \(\reals ^2\).
	Now since \(0\) is the additive identity of \(\reals\), by the
	definition of subtraction over \(\reals\), we know that \(r - r = 0\).
	Observe that \(0\) is an integer. By the definition of \(\theset{S'}\),
	we see that \(\thepair{r}{r}\) is in \(\theset{S'}\). Therefore
	\(\theset{S'}\) is reflexive by definition.
\end{proof}

\begin{thm}\thlabel{s03p05at2}
	\(\theset{S'}\) is symmetric.
\end{thm}
\begin{proof}
	Assume \(\theset{S'}\) is the set of all \(\thepair{x}{y}\) in
	\(\reals ^ 2\) such that \(y - x\) is an integer. Assume
	\(\thepair{a}{b}\) is in \(\theset{S'}\). Note that \(a \isin \reals\),
	that \(b \isin \reals\), and that there exists some \(k \in \reals\)
	such that \(b - a = k\) and \(k\) is known to be an integer. Now by
	algebraic manipulation of the equation \(b - a = k\), we have that
	\(b = k + a\), and that \(0 = k + a - b\), and finally that
	\(-k = a - b\). By the symmetry of \(=\) in \(\reals\), we have that
	\(a - b = -k\). As we have already established that \(k\) is an integer,
	we know that \(-k\) is also an integer by the definition of the
	integers. So \(\thepair{b}{a}\) is in \(\theset{S'}\) by definition.
	Therefore, by definition, \(\theset{S'}\) is symmetric.
\end{proof}

\begin{thm}\thlabel{s03p05at3}
	\(\theset{S'}\) is transitive.
\end{thm}
\begin{proof}
	Assume \(\theset{S'}\) is the set of all \(\thepair{x}{y}\) in
	\(\reals ^ 2\) such that \(y - x\) is an integer. Assume both
	\(\thepair{a}{b}\) and \(\thepair{b}{c}\) are in \(\theset{S'}\). Note
	that \(a \isin \reals\), that \(b \isin \reals\), that
	\(c \isin \reals\), that there exists some \(k \in \reals\) such that
	\(b - a = k\), that there exists some \(j \in \reals\) such that
	\(c - b = j\), and that both \(k \isin \integers\) and
	\(j \isin \integers\). Now by algebraic manipulation of the equation
	\(b - a = k\), we have that \(b = k + a\). By substitution into the
	equation \(c - b = j\), we have that \(c - (k + a) = j\), which can be
	simplified to \(c - k - a = j\) by distribution. By algebraic
	manipulation, we have that \(c - a = j + k\). Note that by the closure
	of addition over the integers, we know that there exists some
	\(m \in \integers\) such that \(m = j + k\). By substitution, we have
	that \(c - a = m\). So \(\thepair{a}{c}\) is in \(\theset{S'}\) by
	definition. Therefore, by definition, \(\theset{S'}\) is transitive.
\end{proof}

\begin{thm}\thlabel{s03p05at4}
	\(\theset{S'}\) is an equivalence relation on \(\reals\).
\end{thm}
\begin{proof}
	As \(\theset{S'}\) is reflexive (by \thref{s03p05at1}), symmetric (by
	\thref{s03p05at2}), and transitive (by \thref{s03p05at3}), we can
	therefore say that  \(\theset{S'}\) is an equivalence relation by
	definition.
\end{proof}

\begin{thm}\thlabel{s03p05at5}
	\(\theset{S}\) is a subset of \(\theset{S'}\).
\end{thm}
\begin{proof}
	Assume \(\theset{S'}\) is the set of all \(\thepair{x}{y}\) in
	\(\reals ^ 2\) such that \(y - x\) is an integer. Assume \(\theset{S}\)
	is the set of all \(\thepair{x}{y}\) in \(\reals ^ 2\) such that
	\(y = x + 1\) and \(0 < x < 2\). Let \(\thepair{a}{b}\) be in
	\(\theset{S}\). Note that \(a \isin \reals\), that \(b \isin \reals\),
	that \(b = a + 1\), that \(a > 0\), and that \(a < 2\). By algebraic
	manipulation of the equation \(b = a + 1\), we have that \(b - a = 1\).
	Observe that \(1\) is an integer. So \(\thepair{a}{b}\) is in
	\(\theset{S'}\) by definition. Therefore \(\theset{S}\) is a subset of
	\(\theset{S'}\).
\end{proof}

\begin{remark}
	The equivalence classes of \(\theset{S'}\) are the sets of all values in
	\(\reals\) that differ from each other by an integer. By choosing the
	unique representative \(r \in \reals\) in each equivalence class such
	that \(0 \leq r < 1\), we can see that while each equivalence class has
	only a countably infinite number of members, there is an uncountably
	infinite number of equivalence classes in \(\theset{S'}\).
\end{remark}


\end{document}
