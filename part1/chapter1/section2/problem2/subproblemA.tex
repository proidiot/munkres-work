\documentclass[main.tex]{subfiles}

\begin{document}

\subproblem{}\label{s2p2a}

Show that function inversion preserves inclusion. Symbolically, that is:
\[B_0 \subset B_1 \implies \inverse{f}(B_0) \subset \inverse{f}(B_1)\]

\begin{thm}
	The subset relation is preserved by the preimage of a function.
	Symbolically, given \(\mapping{f}{A}{B}\) and \(B_0 \subset B\) and
	\(B_1 \subset B\), that is:
	\[B_0 \subset B_1 \implies \inverse{f}(B_0) \subset \inverse{f}(B_1)\]
\end{thm}
\begin{proof}
	Let \(A\) and \(B\) be sets, let \(f\) be a mapping from \(A\) to \(B\),
	let \(B_0\) be a subset of \(B\), let \(B_1\) also be a subset of \(B\).
	Assume \(B_0\) is a subset of \(B_1\), and assume \(x\) is in
	\(\inverse{f}(B_0)\). By the definition of preimage, we know there must
	be some value \(y = f(x)\) in \(B_0\). As we have established that
	\(B_0\) is a subset of \(B_1\), we know that \(y\) must also be a member
	of \(B_1\). Recall that the definition of preimage of \(B_1\) under
	\(f\) gives the set of all values \(a\) such that \(f(a)\) is in
	\(B_1\). So since \(y = f(x)\) is in \(B_1\), then we can say \(x\) is
	in \(\inverse{f}(B_1)\). Thus \(\inverse{f}(B_0)\) is a subset of
	\(\inverse{f}(B_1)\). Therefore \(B_0\) being a subset of \(B_1\)
	implies that \(\inverse{f}(B_0)\) is a subset of \(\inverse{f}(B_1)\).
\end{proof}

\end{document}
