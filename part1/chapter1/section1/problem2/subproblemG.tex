\documentclass[main.tex]{subfiles}

\begin{document}

\subproblem{}\label{2g}
\[A \cap (B \setminus C) = (A \cap B) \setminus (A \cap C)\]
\begin{thm}[Equality holds in~\ref{2g}]
	Given sets \(A\), \(B\), and \(C\), the intersection of \(A\) and the
	portion of \(B\) which does not overlap with \(C\) is equal to the
	portion of the intersection of \(A\) and \(B\) without the portion
	overlapping the intersection of \(A\) and \(C\). Symbolically, that is:
	\[A \cap (B \setminus C) = (A \cap B) \setminus (A \cap C)\]
\end{thm}
\begin{proof}
	For equality among sets to hold, it is necessary that both sets are
	demonstrably subsets of each other. Let \(A\), \(B\), and \(C\) be sets.

	\medskip
	First, let's consider the intersection of \(A\) and the portion of \(B\)
	which does not overlap with \(C\). Since the empty set is already known
	to be a subset of all sets, in order to demonstrate this first subset
	relationship it is only necessary to demonstrate that if some element
	were indeed in \(A \cap (B \setminus C)\), then it would follow that the
	element must also be in \((A \cap B) \setminus (A \cap C)\).

	Let \(x\) be some arbitrary member of \(A \cap (B \setminus C)\). We
	know that \(x\) must be in both \(A\) and \(B \setminus C\) due to the
	definition of set intersection. We can also say that \(x\) must be in
	\(B\) and that \(x\) must not be in \(C\) given the definition of set
	difference. So since \(x\) must be in both \(A\) and \(B\), we know that
	\(x\) is in \(A \cap B\) given the definition of set intersection. So
	all that remains to demonstrate the subset relationship is to show that
	\(x\) is not in \(A \cap C\).

	Let's suppose for a bit that \(x\) really was somehow in \(A \cap C\).
	That would imply that both \(x\) was in \(A\) and that \(x\) was in
	\(C\), again by the definition of set intersection. However, this
	contradicts the fact that \(x\) must not be in \(C\), which we had
	already established before we made this absurd supposition. So it must
	be the case that \(x\) is not in \(A \cap C\). As a result, we can say
	with certainty that the intersection of \(A\) and the portion of \(B\)
	which does not overlap with \(C\) is a subset of the portion of the
	intersection of \(A\) and \(B\) without the portion overlapping the
	intersection of \(A\) and \(C\).

	So we have demonstrated that
	\[A \cap (B \setminus C) \subset (A \cap B) \setminus (A \cap C)\]

	\medskip
	Second, let's consider the portion of the intersection of \(A\) and
	\(B\) without the portion overlapping the intersection of \(A\) and
	\(C\). Again, since the empty set is already known to be a subset of all
	sets, we simply need to demonstrate that if some element were indeed to
	exist in \((A \cap B) \setminus (A \cap C)\), then it would follow that
	the element must also be in \(A \cap (B \setminus C)\).

	Let \(x\) be some arbitrary member of the intersection of \(A\) and
	\(B\) without the portion overlapping the intersection of \(A\) and
	\(C\). We know that \(x\) must be in \(A \cap B\) and that \(x\) must
	not be in \(A \cap C\) due to the definition of set difference. Further,
	we know that \(x\) must be in both \(A\) and \(B\) given the definition
	of set intersection. So since \(x\) is in \(A\), all that remains to
	demonstrate the subset relationship is to show that \(x\) must in
	\(B \setminus C\) in order to show that \(x\) in
	\(A \cap (B \setminus C)\).

	Recall that \(x\) is not in \(A \cap C\). In other words, it is not the
	case that both \(x\) is in \(A\) and \(x\) is in \(C\). Using De
	Morgan's Law, we can translate this statement into the equivalent
	statement that either \(x\) is not in \(A\) or \(x\) is not in \(C\) (or
	both). Since we have already established that \(x\) must be in \(A\), we
	can say that \(x\) must not be in \(C\) using the disjunctive syllogism.
	Combined with our previous demonstration that \(x\) must be in \(B\),
	we can say that \(x\) must be in \(B \setminus C\) using the definition
	of set difference.

	So we have demonstrated that
	\[(A \cap B) \setminus (A \cap C) \subset A \cap (B \setminus C)\]

	\medskip
	Finally, since we have demonstrated that each set is a subset of the
	other, we can say that the sets are equal.
\end{proof}

\end{document}

