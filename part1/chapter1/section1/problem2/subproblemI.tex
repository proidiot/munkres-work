\documentclass[main.tex]{subfiles}

\begin{document}

\subproblem{}\label{2i}
\[(A \cap B) \cup (A \setminus B) = A\]
\begin{thm}[Equality holds in~\ref{2i}]
	Given sets \(A\) and \(B\), the union between the intersection of \(A\)
	and \(B\) with the portion of \(A\) not overlapping \(B\) is equal to
	\(A\). Symbolically, that is
	\[(A \cap B) \cup (A \setminus B) = A\]
\end{thm}
\begin{proof}
	Let \(A\) and \(B\) be sets. We will demonstrate equality by
	demonstrating that \((A \cap B) \cup (A \setminus B)\) and \(A\) are
	subsets of each other.

	First, we will address \((A \cap B) \cup (A \setminus B)\). Since the
	empty set is already known to be a subset of all sets, we will simply
	examine whether some arbitrary value which might exist in the set would
	necessarilly also be in \(A\). So let \(x\) be in
	\((A \cap B) \cup (A \setminus B)\). We know that \(x\) is either in
	\(A \cap B\) or in \(A \setminus B\) due to the definition of set union.
	In the case that \(x\) is in \(A \cap B\), we would know that \(x\) is
	in \(A\) and \(x\) is in \(B\) by the definition of set intersection,
	and so we immediately have that \(x\) is necessarilly in \(A\). On the
	other hand, if \(x\) is in \(A \setminus B\), then \(x\) is in \(A\) and
	\(x\) is not in \(B\) given the definition of set difference. Since we
	have shown that such an \(x\) is in \(A\) no matter what, it is certain
	that \((A \cap B) \cup (A \setminus B)\) is a subset of \(A\).

	Second, we will proceed from \(A\). Again, since the empty set is known
	to be a subset of all sets, we only need to consider whether some
	arbitrary value which might exist in \(A\) would necessarilly also exist
	in \((A \cap B) \cup (A \setminus B)\). Let \(x\) be in \(A\). Notice
	that it is either the case that \(x\) is also in \(B\) or it is not the
	case that \(x\) is in \(B\). If \(x\) is indeed in \(B\), then the fact
	that \(x\) is in both \(A\) and \(B\) means that \(x\) is in
	\(A \cap B\) by the definition of set intersection, and by disjunctive
	introduction we know that \(x\) is also in
	\((A \cap B) \cup (A \setminus B)\). On the other hand, if \(x\) is not
	in \(B\), then the definition of set difference tells us that \(x\) is
	in \(A \setminus B\), and by extension \(x\) is also in
	\((A \cap B) \cup (A \setminus B)\). So \(A\) is a subset of
	\((A \cap B) \cup (A \setminus B)\).
\end{proof}

\end{document}

