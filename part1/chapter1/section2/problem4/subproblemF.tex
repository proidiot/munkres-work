\documentclass[main.tex]{subfiles}

\begin{document}

\subproblem{}\label{s2p4f}

Summarize your answers to \(\ref{s2p4b} \thru~\ref{s2p4e}\) in the form of a
theorem.

\begin{thm}\thlabel{s2p4ft1}
	Given functions \(\mapping{f}{A}{B}\) and \(\mapping{g}{B}{C}\), if
	\(f\) is injective and the restrictuon of \(g\) to the image of \(A\)
	under \(f\) is also injective, then \(\compose{g}{f}\) is injective.
\end{thm}
\begin{proof}
	Let \(A\), \(B\), and \(C\) be sets, let \(f\) be an injective mapping
	from \(A\) to \(B\), and let \(g\) be a mapping from \(B\) to \(C\) such
	that \(\restrict{g}{f(A)}\) is injective.

	Assume \(x\) and \(x'\) are in \(A\) such that \(x \neq x'\). We know
	that \(f(x) \neq f(x')\) by the contraposition of the definition of
	the injectivity of \(f\). Further, since \(f(x)\) and \(f(x')\) are in
	\(f(A)\) by the definition of image under a function, we know that
	\(g(f(x)) \neq g(f(x'))\) by the contraposition of the definition of
	the injectivity of \(\restrict{g}{f(A)}\). By the definition of function
	composition, we know that
	\(\composeapply{g}{f}{x} \neq \composeapply{g}{f}{x'}\). Thus, as we
	have shown that \(x \neq x'\) implies
	\(\composeapply{g}{f}{x} \neq \composeapply{g}{f}{x'}\), we know that
	\(\composeapply{g}{f}{x} = \composeapply{g}{f}{x'}\) implies \(x = x'\)
	by contraposition. Therefore if \(f\) and \(\restrict{g}{f(A)}\) are
	injective, then \(\compose{g}{f}\) is injective.
\end{proof}

\begin{thm}\thlabel{s2p4ft2}
	Given functions \(\mapping{f}{A}{B}\) and \(\mapping{g}{B}{C}\), if the
	restrictuon of \(g\) to the image of \(A\) under \(f\) is surjective,
	then \(\compose{g}{f}\) is surjective.
\end{thm}
\begin{proof}
	Let \(A\), \(B\), and \(C\) be sets, let \(f\) be a mapping from \(A\)
	to \(B\), and let \(g\) be a mapping from \(B\) to \(C\) such that
	\(\restrict{g}{f(A)}\) is surjective.

	Assume \(z\) is in \(C\). We know that there exists \(y\) in \(f(A)\)
	such that \(z = g(y)\) due to the definition of surjectivity. Further,
	we know there exists \(x\) in \(A\) such that \(y = f(x)\) by the
	definition of the image of a set under a function. By substitution, we
	see that \(z = g(f(x))\). Finally, by the definition of function
	composition, we see that \(z = \composeapply{g}{f}{x}\). Thus if \(z\)
	is in \(C\) we can say there exists \(x\) in \(A\) such that
	\(z = \composeapply{g}{f}{x}\). Therefore if \(\restrict{g}{f(A)}\) is
	surjective, then \(\compose{g}{f}\) is surjective.
\end{proof}

\begin{thm}
	Given functions \(\mapping{f}{A}{B}\) and \(\mapping{g}{B}{C}\), we can
	say that \(\compose{g}{f}\) is injective iff \(f\) is injective and the
	restrictuon of \(g\) to the image of \(A\) under \(f\) is also
	injective.
\end{thm}
\begin{proof}
	Let \(A\), \(B\), and \(C\) be sets, let \(f\) be a mapping from \(A\)
	to \(B\), and let \(g\) be a mapping from \(B\) to \(C\).

	First, assume \(\compose{g}{f}\) is injective. By \thref{s2p4ct1}, we
	have that \(f\) is injective. Also, by \thref{s2p4ct3}, we have that
	the restriction of \(g\) to the image of \(A\) under \(f\) is injective.
	Thus by conjunction, we have that \(f\) is injective and
	\(\restrict{g}{f(A)}\) is injective.

	Next, assume \(f\) is injective and \(\restrict{g}{f(A)}\) is injective.
	Thus by \thref{s2p4ft1}, we have that \(\compose{g}{f}\) is injective.

	As we have demonstrated that each implies the other, we have therefore
	demonstrated that \(\compose{g}{f}\) is injective iff \(f\) is injective
	and \(\restrict{g}{f(A)}\) is injective.
\end{proof}

\begin{thm}
	Given functions \(\mapping{f}{A}{B}\) and \(\mapping{g}{B}{C}\), we can
	say that \(\compose{g}{f}\) is surjective iff the restrictuon of \(g\)
	to the image of \(A\) under \(f\) is surjective.
\end{thm}
\begin{proof}
	Let \(A\), \(B\), and \(C\) be sets, let \(f\) be a mapping from \(A\)
	to \(B\), and let \(g\) be a mapping from \(B\) to \(C\).

	First, assume \(\compose{g}{f}\) is surjective. Thus by \thref{s2p4et1},
	we have that \(\restrict{g}{f(A)}\) is also surjective.

	Next, assume \(\restrict{g}{f(A)}\) is surjective. Thus by
	\thref{s2p4ft2}, we have that \(\compose{g}{f}\) is surjective.

	As we have demonstrated that each implies the other, we have therefore
	demonstrated that \(\compose{g}{f}\) is surjective iff
	\(\restrict{g}{f(A)}\) is surjective.
\end{proof}

\end{document}
