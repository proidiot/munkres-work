\documentclass[main.tex]{subfiles}

\begin{document}

\subproblem{}\label{2f}
\[A \setminus (B \setminus A) = A \setminus B\]
\begin{thm}[Equality fails in~\ref{2f}]
	Given sets \(A\) and \(B\), it is not the case that the portion of \(A\)
	without the portion of \(B\) which does not overlap with \(A\) is equal
	to the portion of \(A\) which does not overlap with \(B\). Symbolically,
	that is:
	\[\neg [A \setminus (B \setminus A) = A \setminus B]\]
\end{thm}
\begin{proof}
	Choose set \(A\) to be the set of integers. Choose set \(B\) to be the
	set of all odd integers specificaly. Since the set of all odd integers
	only includes members which are also in the set of all integers,
	removing \(A\) from \(B\) leaves the empty set. That is, for our
	particular choices of \(A\) and \(B\), we have that
	\(B \setminus A = \emptyset\). By substitution, removing from \(A\) the
	portion of \(B\) that does not overlap with \(A\) is simply removing the
	empty set from \(A\). Removing the empty set leaves a set unchanged, so
	our particular choices for \(A\) and \(B\) actually give
	\(A \setminus (B \setminus A) = A\).

	Continuing with our previous choices for sets \(A\) and \(B\), we can
	see that removing the the set of all odd integers (i.e., \(B\)) from
	the set of all integers (i.e., \(A\)) leaves the set of all even
	integers. Now suppose it really was the case that the portion of \(A\)
	without the portion of \(B\) which does not overlap with \(A\) is equal
	to the portion of \(A\) which does not overlap with \(B\). Then that
	would mean that the set of all integers is equal to the set of just the
	even integers, which is absurd as there are certainly integers which are
	not even (specifically, there are odd integers). Since we have reached a
	contradiction, we can say that we were incorrect in our supposition, and
	so it is not always the case that \(A\) without the portion of \(B\)
	which does not overlap with \(A\) is equal to \(A\) without the
	overlapping portion of \(B\).
\end{proof}

\begin{thm}[RHS is a subset of LHS in~\ref{2f}]
	Given sets \(A\) and \(B\), the portion of \(A\) which does not overlap
	with \(B\) is a subset of the portion of \(A\) without the portion of
	\(B\) which does not overlap with \(A\). Symbolically, that is:
	\[A \setminus B \subset A \setminus (B \setminus A)\]
\end{thm}
\begin{proof}
	Assume there are two arbitrary sets \(A\) and \(B\). Note that removing
	the portion of \(B\) which overlaps with \(A\) is equivalent to taking
	the intesection of \(B\) and the complement of \(A\). Now removing from
	\(A\) the intersection of \(B\) and the complement of \(A\) would only
	remove portions of \(A\) that were also in the complement of \(A\) which
	also happened to be in \(B\). As \(A\) and the complement of \(A\)
	necessarilly do not overlap at all, this would remove nothing from \(A\)
	regardless of the nature of \(B\). Finally, taking the portion of \(A\)
	which does not overlap with \(B\) gives a subset of \(A\), and, by
	substitution, a subset of the portion of \(A\) which does not overlap
	with the portion of \(B\) which does not itself overlap with \(A\).
\end{proof}

\end{document}

