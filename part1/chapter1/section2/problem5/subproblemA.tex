\documentclass[main.tex]{subfiles}

\begin{document}

\subproblem{}\label{s2p5a}

Show that if \(f\) has a left inverse, \(f\) is injective; and if \(f\) has a
right inverse, \(f\) is surjective.

\begin{thm}\thlabel{s2p5at1}
	The identity function is injective.
\end{thm}
\begin{proof}
	Let \(i_C\) be the identity function of a set \(C\). Assume \(x\) and
	\(x'\) are in \(C\) such that \(i_C(x) = i_C(x')\). By the definition of
	the identity function, we see that \(x = i_C(x) = i_C(x') = x'\). So
	\(i_C(x) = i_C(x')\) implies \(x = x'\), and therefore \(i_C\) is
	injective.
\end{proof}

\begin{thm}\thlabel{s2p5at2}
	The identity function is surjective.
\end{thm}
\begin{proof}
	Let \(i_C\) be the identity function of a set \(C\). Assume \(x\) is in
	\(C\). By the definition of the identity function, we know that
	\(x = i_C(x)\). So given \(x\) in the co-domain of \(i_C\), we see there
	exists \(x\) in the domain of \(i_C\) such that \(x = i_C(x)\).
	Therefore \(i_C\) is surjective.
\end{proof}

\begin{thm}
	A function with a left inverse is injective.
\end{thm}
\begin{proof}
	Let \(A\) and \(B\) be sets, let \(f\) be a mapping from \(A\) to \(B\),
	and let \(g\) be a left inverse of \(f\). By the definition of left
	inverse, we know that \(\compose{g}{f} = i_A\). Since \(i_A\) is
	injective by \thref{s2p5at1}, we can say that \(\compose{g}{f}\) is
	injective.  Finally, by \thref{s2p4ct1}, we know that \(f\) is
	injective.
\end{proof}

\begin{thm}
	A function with a right inverse is surjective.
\end{thm}
\begin{proof}
	Let \(A\) and \(B\) be sets, let \(f\) be a mapping from \(A\) to \(B\),
	and let \(g\) be a rightt inverse of \(f\). By the definition of right
	inverse, we know that \(\compose{f}{g} = i_B\). Since \(i_B\) is
	surjective by \thref{s2p5at2}, we can say that \(\compose{f}{g}\) is
	surjective. By \thref{s2p4et1}, we can say that \(\restrict{f}{g(B)}\)
	is surjective. Finally, since \(g(B)\) is a subset of \(A\), we can also
	say \(f\) is also surjective.
\end{proof}

\end{document}
